\documentclass[a4paper,11pt]{jreport}

\usepackage{times} % use Times Font instead of Computer Modern
\usepackage[dvipdfmx]{graphicx} %  use figure package
\usepackage{amsmath} % use matrix  notation
\usepackage{bm} % use vector notation
\usepackage{listings} % list the code(C,SAD) at appendix
\usepackage{here} % define figure position
\usepackage{siunitx} % use SI of Units
\usepackage[top=30truemm,bottom=30truemm,left=30truemm,right=30truemm]{geometry}  %  use true geometry

\setcounter{tocdepth}{3} % Output subsection to the table of contents.
\setcounter{page}{-1} % reduce page number of a cover
\setlength{\oddsidemargin}{0.1in}
\setlength{\evensidemargin}{0.1in}
\setlength{\topmargin}{0in}
\setlength{\textwidth}{6in}
\setlength{\parskip}{0em}
\setlength{\topsep}{0em}

% Let's start !
\begin{document}

% title 
\begin{titlepage}
\begin{center}
\vspace*{.06\textheight}
{\LARGE 修士論文\par}\vspace{2.5cm} % Department name
{\huge KEK-PFにおける入射率改善のための研究\par}\vspace{10.0cm} % Thesis title 
{\LARGE 広島大学理学研究科物理科学専攻\par}\vspace{0.25cm}
{\LARGE 放射光物理研究室\par}
{\LARGE M170458\par}
{\LARGE 平野 広太 \par}\vspace{1.5cm} % Author name
{\LARGE 主査 島田 賢也\par}
{\LARGE 副査 平谷 篤也\par}
%{\LARGE \today} % Data 
\end{center}
\end{titlepage}
% 概要
\par
\vspace{0pt plus 1fil}
\newpage
\begin{center}
{\Large 修士論文要項\par}
{\large 広島大学大学院 理学研究科 物理科学専攻 放射光物理研究室\par}
{\Large 平野 広太\par}
\end{center}
\section*{はじめに}
高エネルギー加速器研究機構 (KEK)にある蓄積リング型放射光源 (Photon Factory:PF)では、蓄積リングを周回する電子ビームから発生する光  (放射光)を使って物質・生命の構造解析から機能発現の仕組みの解明まで、幅広い研究が行われている。
ユーザーが実験に使う放射光の強さは蓄積リングを周回する電子ビームの電流値に比例するが、電子ビームはガス散乱や量子寿命 (量子力学的効果で大きなエネルギーの放射光を出すことによる反跳)、ビーム内散乱などによって時間と共に失われてゆく。そのため、定期的に入射器 (LINAC)から電子ビームを供給し、輸送路 (Beam Transport:BT)を通してリングまで運搬、リングに注ぎ足している。リングに新しい電子ビームを供給することを入射と呼ぶ。PFを含む一般的な蓄積リングへの入射では、キッカー電磁石とセプタム電磁石という2種類のパルス電磁石が用いられる。それらの電磁石を最適化し、入射に求められる条件、輸送中のビームロスを最小化すること、蓄積リング内に入れた入射ビームを損失なく捕獲、蓄積すること、既存の蓄積ビームに擾乱を与えないこと、を満たすようにしている。

具体的には、セプタム電磁石は輸送路の終端部、蓄積リング直前に設置され、入射ビームのみの軌道を曲げ、蓄積ビームの設計軌道と向きを揃えるために用いられる。入射ビームだけを蹴り、蓄積ビームに影響を与えないようにするため、蓄積リングの真空ダクトと入射路の間にはセプタム (横隔膜の意)壁という銅板を配置し、セプタムをパルス的に励磁することで、その板状に渦電流を発生させ、磁場を遮蔽している。一方、キッカー電磁石は蓄積リング内に配置され、蓄積ビームに対しては入射点にバンプ (局所的な軌道変化)を形成する。入射点より上流に2台、下流に2台、合計4台のキッカーを使えば、入射点における蓄積ビームの位置と傾きの両方を任意のオプティクスに対して自由に決めることができる。入射点上流のキッカー電磁石は、蓄積ビームのみを蹴り、入射点で入射路側へ軌道を寄せる。入射点における蓄積ビームと入射ビームの間の距離が、蓄積リング内での入射ビームの振動の初期振幅となる。入射点下流のキッカー電磁石は、蓄積ビームと入射ビームの両者を同時に同じだけ蹴るが、蓄積ビームに対しては軌道変位を打ち消して中心軌道に戻すように、入射ビームに対してはリング外側にいたものを蓄積リングダクト内に納めるように、それぞれ軌道を曲げる。蓄積ビームの軌道は入射の瞬間だけ、入射点近傍だけで変化させるが、そのような操作を局所 (ローカル)パルスバンプという。入射ビームは中心軌道から離れた場所に打ち込まれるため、蓄積リングを周回しながら大振幅の振動 (ベータトロン振動)を始める。蓄積リングでは放射光を出す効果でそのような振動は数千ターンの時定数で減衰 (放射減衰)してゆくので、入射ビームは周回しながらだんだん振幅を減らし、やがて、既存の蓄積ビームと一体となる。

蓄積リング内の電子ビームは、電磁石による線形の収束力による、変調された調和振動 (ベータトロン振動)を行っている。ベータトロン振動数のエネルギー収差を補正するためには、非線形の磁場が必要であり、そのせいで振幅が大きくなると振動が共鳴振動になって粒子が失われる。リング中心軌道付近のベータトロン振動の安定領域をダイナミックアパーチャという。入射ビームは真空ダクトで決まる物理的なアパーチャと、力学から決まるダイナミックアパーチャの両方の内側に入れる必要がある。逆に、蓄積ビーム内で入射ビームが失われる原因は、物理的にダクトに衝突することと、共鳴振動で振動振幅が大きくなることである。入射ビームの振動の振幅を減らすには入射点で蓄積ビームのバンプを大きくすれば良いが、最小値はビーム自身のサイズとセプタム壁の厚さで決まる。また、入射ビームの位置や傾きの誤差は振幅の増大につながり、蓄積ビームのバンプの誤差はリング全体で蓄積ビームを揺らすことになる。また、入射バンプは周回周期に合わせて1回だけ蹴られるものであるが、入射されるタイミングのビームに対してはそうでも、蓄積ビームには電子の塊 (バンチ)が連続的に複数存在 (マルチバンチ運転という)し、キッカー電磁石の立ち上がりから立ち下がりまで、断続的に電子がやってくる。入射タイミングのビームだけに対する振動と入射率の最適化と、マルチバンチ全体に対する誤差振動低減の最適化は、理想的な場合は同じであるが、現実的には様々な誤差の影響で、必ずしも両立しない。最適化には既存の入射に関するパラメータを正確に知り、制御する必要があるが、PFでは、経年変化や震災などの影響でそれが困難であった。本研究は、現状の入射パラメータの精密測定と、その結果を使った、入射の最適化、特にPFのユーザー運転で現在問題となっているマルチバンチ全体に対する入射振動の低減を行ったものである。

     \begin{figure}[H]
         \centering
         \includegraphics[width=130mm]{./figure/00_Foreword/scheme_Injection.png}
         \caption{キッカー入射の概要図}
     \end{figure}

\section*{本研究の目的}
PFリングでは、長年の運転で積み重なった誤差の影響と東日本大震災の影響で、入射率の低下と入射中の蓄積ビームの揺らぎが問題となっていた。入射 (捕獲)率の低下は蓄積リング内でのビームロスによる放射線の発生につながり、場合によっては実験ホールの一部に立ち入り制限がかかることとなる。また、蓄積ビームのバンプ誤差によるリング全体での振動は、ユーザー実験中の放射光強度の変動を引き起こし、ユーザーのデータ取得を中断させる。そこで本研究で、入射率の改善を目的に、最初に、現在の入射に関するパラメータ、すなわち、入射点における入射ビームの位置 (と向き)、キッカー電磁石のビームに対する実際の励磁波形の測定を行い、次に、それらを使って実際の最適化、蓄積ビームの振動低減を実施した。
入射路や入射点近傍にはビームの位置を測定するモニタが設置されているが、モニタ自体の位置が較正されておらず、絶対値としての位置が不明であった。モニタを含むセプタム電磁石と真空槽は1980年代製であり、度重なるリング改造や震災の影響で、正確な位置が分からず、真空を開封して測定するわけにもゆかず、経験則での最適化が積み重ねられてきた。キッカーに関しては、実験室での磁場測定のデータはあるものの、リング内設置後の磁場は環境や配線、配置の影響で実験室とは違ってしまっていることが分かっており、過去にスタディが行われたこともあるが、そのデータを使っての最適化までは至っていない。本研究では、蓄積リング内のビーム位置モニタを使い、それらの不確定な情報を実ビームを使って測定することで決定的なデータとして確立、それを使って最適化までを行った。一般的に、電子ビームの位置モニタは複数回の平均で精度を上げており、小電荷ビームを1回通過しただけのデータで位置を正確に計測することが難しい。生の電圧の時間波形を独自に解析し、それを可能にしたことも本研究の特徴である。

\section*{入射に関するパラメータの測定}
はじめに、現在の入射状況を調べるため、入射に関するパラメータの測定を行った (4章)。具体的には、入射ビームの位相空間情報 (入射点での位置と傾き)、キッカー電磁石の磁場パルス波形、セプタム電磁石、ビーム輸送路の補正電磁石の応答、である。その測定を元に、現在の入射に関する各種パラメータを推定した。

加速器は多数の部品で構成される複雑なシステムであり、電磁石の設置誤差や磁場誤差、裾磁場の影響や挿入光源の影響など、理想的なシミュレーションには含まれない多くの誤差を含む。また、入射キッカーの場合、PFリングでは周長が短いため、バンプ内に非線形磁場 (振幅自乗の蹴り角)を発生させる6極電磁石が配置されているが、非線形磁場は入射タイミング以外のタイミングの電子ビームに対してはバンプを閉じさせない原因となる。また、2周回周期 (1.2 \si{\mu sec})幅の半正弦波形のパルス励磁ということから、高電圧大電流が必要であり、過渡現象、ダクトや周辺環境での渦電流の効果、電源や電磁石本体の個体差など、考慮すべき誤差が非常に多い。全ての誤差がそれなりに優勢である一方、個別の誤差を独立に調べて補正することは極めて困難である。そこで、実ビームを使って測定を行えば、全ての誤差の影響を含んだ正しい値を決めることができる。測定は蓄積リング入射点下流の長直線部両端のBPM (Beam Position Monitor)電極に信号検出回路を接続して行った。BPMはビームの位置情報しか得られないが、2点で測定することで、位置と傾きに変換することができる。また、蓄積リング内の電子の運動 (ベータトロン振動の軌跡)はリングのオプティクスで決まるので、ある点で位置と傾きが分かれば、上下流の状態を知ることができる。

     \begin{figure}[H]
         \centering
         \includegraphics[width=120mm]{./figure/00_Foreword/scheme_BBM.png}
         \caption{ビームベース測定の概要図}
     \end{figure}

キッカー電磁石のパルス波形の測定は、蓄積ビームに対する正確なバンプを形成するために必要な情報である。測定は、蓄積ビームにバンチを1個だけ蓄積 (シングルバンチモード)し、キッカーを個別に励磁、励磁のパルスタイミングを変えながら、BPMでビームの測定を行うことで、キッカーは系全体の波形の再構築を行った。繰り返しになるが、キッカー磁場波形や磁場強度は、加速器設置前に行われた実験室での磁場測定の結果を元に決められているが、設置後の変化が大きいため、ビームを使っての測定が必須となっている。 (実際に磁場測定結果を基に最適化しても大きな誤差が残るし、ビームで測定した波形も実験室測定の結果とは異なっていた。) 測定結果によると、4台のキッカーの電磁石の励磁特性には個体差があることが分かった。波形は半正弦的な波形に反射が続くような波形になるが、半正弦的部分の周期に関しても揃っておらず、どんなに最適化してもマルチバンチ全体に対して振動をゼロにすることが困難であることが分かった。


      \begin{figure}[H]
         \centering
         \includegraphics[width=120mm]{./figure/00_Foreword/compare_KK.png}
         \caption{磁場測定とビームベースで測定したキッカーパルス波形の比較。実線は加速器設置前に磁場測定した結果を表す。プロットは加速器設置後にビームベースで測定した結果を表す。加速器設置後にパルス波形が変わることを理解できる。}
     \end{figure}

蓄積リングのビーム位置モニターは多数回の平均処理によって精度を向上させており、BTのモニタはそれがないために精度が落ちる。リングのモニタの精度は数ミクロン、BTでは数百ミクロンである。今回の測定では、蓄積リングでの測定においても、キッカーで蹴った直後、入射ビームの入射直後の位置情報が必要であり、平均処理を行うことができなかった。また、LINACは本質的に不安定であり、入射ビームのパルス繰り返しには、大きなエネルギー、電荷、位置のばらつきが伴っている。電荷は0~1\si{nC}まで、位置は入射点で数\si{mm}パルスごとに不安定である。そのようなもとで精度良く位置を測定するためには、既存の検波回路とデータ解析ソフトウェアを使うことはできなかった。BPMはリングダクトに絶縁したボタン電極を配置すると、電極上に電極からビームまでの距離に応じた電圧が誘起される現象を利用してビーム位置を測定するものである。今回の実験では、電極から返される生の電圧波形に、LINACからの電荷量に対する例外処理 (電荷が十分に来ていない時はデータを使わない)を加え、電極からの距離の3次の非線形係数までを考慮し、フィッティングを行った。非線形係数はBPM設置前にワイヤスキャンで測定された較正値値を利用した。通常の蓄積状態の値をオフセットとして考えることで、リング中心軌道からの相対的な距離を推定することができる。電圧波形からビームのタイミングだけを抜き出し、ノイズだらけのきれいでない波形をフィッティングして電圧情報を抜き出し、非線形係数を考慮して4電極間の電圧比較からその場所でのビームの位置情報を計算するプログラムは自作のものである。
蓄積リング内のBPMを使った、入射ビームに関する位相空間情報 (入射点での位置と傾き)の推定からは、セプタム壁の位置と入射ビーム、蓄積ビームの位置関係を明らかにすることができた。これはPFリングで将来的に予定されているセプタム電磁石の更新に伴う間接的な測量結果ともよく一致している。また、ばらつきに関しても、スクリーンモニタ (蛍光板モニタ)で目視したジッタとよく一致している。

      \begin{figure}[H]
         \centering
         \includegraphics[width=100mm]{./figure/00_Foreword/plot_injection}
         \caption{入射ビームの位相空間プロット}
     \end{figure}

     \begin{table}[H]
     \caption{入射点での入射ビームの相対位相空間情報}
     \centering
     \begin{tabular}{ l c c } \hline
     & Position [mm] & Angle [mrad] \\ \hline
     Horizontal & $30.5\pm3.20$ & $3.11\pm0.42$ \\
     Vertical    & $0.06\pm0.23$ & $1.02\pm0.23$ \\ \hline
     \end{tabular}
     \end{table}

\section*{入射シミュレーション}
次に、実際に測定した結果に基づき、入射シミュレーションを行った(第5章)。現在、PFリングでは、入射点直上流のセプタム電磁石の真空中冷却水配管にリークがあり、冷却水を使うことができていない。その為、セプタム上流に放射光アブソーバが挿入されており、それが蓄積ビームに対する入射バンプの高さを制限している。また、キッカー電源K2に不具合があり (小さな蹴り角で安定に励磁できない)、使うことができていない。その現状を反映させるために、シミュレーションでは最上流のキッカーK1を固定とし、K2はなし、入射点下流のK3、K4を自由としてビーム振動や入射率に関するシミュレーションを行った。過去にキッカー4台を使ったシミュレーションの結果はあるが、現状をきちんと反映したシミュレーションは初めてである。
バンプ軌道の計算は、入射タイミングのシングルバンチに対して、蹴り角の組み合わせを解析的に求める方法と、マルチバンチの蓄積ビームに対して、バンチ全体にわたる振動量を最小化して求める方法の2通りをまとめた。はじめに、理想的な状況のシングバンチ、シングルターンキックについて計算を行い、次に測定したキッカーのパルス波形を使って誤差による蹴りを含むマルチターンキックへの展開を行った。シングルバンチに対しての最適化は解析的に求められるが、マルチバンチ全体にわたる最適化は同じ手法で行うことはできない。そこで、マルチバンチの蓄積ビーム全体にわたる振動量を評価関数として、その最小値を求める計算から最適化を行う方法をまとめた。
マルチバンチの場合、蓄積リングBPMから観測される情報は、全280個のマルチバンチ全体の平均値である。PFリングは312個のバンチを蓄積可能であるが、実際にはビーム不安定性回避のため、280個の連続バンチの後、32バンチ分は電子がない状態を作っている。ビームに対してK1、K3、K4の3台のタイミングを同時に変えていく場合、タイミングを1周期変えても同じデータとなるはずである。データの確認として、そのようなデータの測定も行った。

     \begin{figure}[H]
         \centering
         \begin{tabular}{c}
         \begin{minipage}{0.45\hsize}
         \centering
         \includegraphics[clip, width=1.0\columnwidth]{./figure/00_Foreword/compare_beam.png}
         \hspace{16mm} [1] actual
         \end{minipage}
         \begin{minipage}{0.45\hsize}
         \centering
         \includegraphics[clip, width=1.0\columnwidth]{./figure/00_Foreword/compare_calc.png}
         \hspace{16mm} [2] model
         \end{minipage}
         \end{tabular}
         \caption{蓄積ビームの振動量の変化。カラーバーの範囲は、蓄積ビームの振動量の変化量を最大最小で正規化した範囲を表す。図の比較から蓄積ビームの振動を与える影響を取り入れて再現できたことを理解できる。}
     \end{figure}

入射 (捕獲)率の見積もりでは、測定した入射ビームの入射点での初期位置と傾き、LOCOという手法で測定したリングオプティクスを使った。LOCOとは、リング補正電磁石の応答行列からオプティックスをフィッティングによって得る手法である。この応答行列の解析に必要なプログラムはKEKの加速器シミュレーションコードSADで自作した。また、応答測定も実際のビームで測定を行った。ガウス分布の多粒子で構成した入射ビームを蓄積リング内に打ち込み、現在のPFの状態で何周回安定に回るかを計算した。結果は、LINACからの電荷量とリング蓄積電流値の増加量から推定される捕獲率とほぼ一致した。

     \begin{figure}[H]
         \centering
         \includegraphics[width=100mm]{./figure/00_Foreword/injeff_mul03.png}
         \caption{多粒子追跡によって100ターン後も生き残る粒子数。カラーバーのレンジは入射率(生存粒子数/生成粒子数=1000)を表す。横軸のK3とK4は下流のキッカーの蹴り角の組み合わせを表す。入射率をよくする方向は、蓄積ビームの振動を犠牲にしたキッカージャンプの効果を表す。現在の入射率はLINACばらつきを考慮して5割前後で揺らいでおり、現在の入射状況を理解できる。}
     \end{figure}

\section*{入射時の蓄積ビーム振動の最適化}
最後に、蓄積ビームの振動を抑制するため、ビームベースで最適化を行った(6章)。現在、PFリングでは、入射時の蓄積ビーム振動が問題となっており、実際にユーザーからは入射の度に光強度が20\%減少するとの報告がなされている。また、入射率だけを問題とするなら、入射ビームの振動を小さくする代わりに蓄積ビームを振動させる方法 (キッカージャンプといい、実際にPF-ARでは利用されていた)も存在する。蓄積電流を一定に保つトップアップ入射においては蓄積ビームに擾乱を与えないことが非常に重要であるため、まずはその最適化を行った。バンプをつくらないことが最適結果にならないよう、ここでもK1は固定とし、K2はなし、K3、K4の蹴り角とタイミングの最適値を求めることとした。その結果、バンプハイトを保ったまま、蓄積ビームの撹乱を抑えた。最適化後、入射が可能である (ユーザー運転時のパラメータの入射率と遜色ない)ことも確認した。
シミュレーションでは全BPMからの振幅データを入射後の数周分用いたが、実際のPFでは、高速BPMは限られている。その為、リング6箇所のBPMを使い、入射後、200周から800周までのデータを評価関数として採用した。評価関数自体の測定プログラム、最適化の自動プログラムは独自に作成したものである。適切な評価関数とパラメータ範囲、測定方法、アルゴリズムなどが設定されて、はじめてパラメータ間の相互作用を無視しての最適化ができる。ここでの大きな課題は応答に含まれるノイズ処理と発散しないための初期設定である。それらを実際にスタディで確立、自動での最適化まで成功させることができた。また実測パラメータを使ったシミュレーション結果と比較しても、最適化結果は十分に計算予想と一致することが分かった。入射という正確な予想、最適化が困難な操作において、実ビームを使った測定から妥当なモデルを作成し、実際の最適化まで自動できちんと収束させたことが本研究の成果である。

     \begin{figure}[H]
         \centering
         \begin{tabular}{c}
         \begin{minipage}{0.45\hsize}
         \centering
         \includegraphics[clip, width=1.0\columnwidth]{./figure/00_Foreword/ori_single_bunch.png}
         \hspace{16mm} [1] original
         \end{minipage}
         \begin{minipage}{0.45\hsize}
         \centering
        \includegraphics[clip, width=1.0\columnwidth]{./figure/00_Foreword/opt_single_bunch.png}
         \hspace{16mm} [2] optimized
         \end{minipage}
         \end{tabular}
         \caption{最適化前後の蓄積ビームの振動量の変化}
     \end{figure}



\section*{結論}
本論文で実施した入射スタディは、従来の問題とされていた誤差を多く含む加速器モデルでのパラメータ補正と最適化を提案した。この手法ではドリフト空間を挟んだBPMの電極さえあれば、コンポーネントの変更なしに同様のスタディを実施することが可能である。本論文はシミュレーションだけでなくビームスタディと合わせた比較、結果をまとめている。また必要な解析コードはすべて独自に開発した。スタディでは測定が困難とされ他施設でも実施されていなかった、輸送路終端部の電磁石の応答と入射ビームの位相空間プロットをビームベースで得た。入射最適化では、発散させることなく自動で振動抑制を成功させている。今後は入射ビームの位相空間調整を行う。また輸送路の電磁石の応答測定、ビームを使ったオプティックスマッチングなど検討する予定である。


% Figure/Table list
\par
\vspace{0pt plus 1fil}
\newpage
\pagenumbering{roman} % I, II, III, IV
\tableofcontents
%\listoffigures
\pagebreak \setcounter{page}{1}
\pagenumbering{arabic} % 1,2,3
%*******************************************************************************
\chapter{序論}
\section{光源加速器}
加速器とは、電子や陽電子、陽子、重陽子、$\alpha$粒子、重イオンなどの粒子のエネルギーを高める装置である。加速器から高速に加速された粒子を原子核固定標的に衝突させることで、粒子の質量、電荷、運動量、スピンなどを測定し、その素粒子の相互作用などを知ることができ、素粒子や原子核などの構造やそれらに働く相互作用を解析するために開発されてきた。粒子のエネルギー加速の基本原理は、電場による加速である。電位差のある電極間に電荷をもった粒子を置くと、ポテンシャルエネルギーに相当するエネルギーを運動エネルギーとして得る。1 \si{V}の電位差間を$1.6\times10^{-19} \si{C}$の電荷が通過した際に得るエネルギーは1 \si{eV}であり、電荷の大きさと電位差のみによって決まる。一つの電極での加速は発生できる高電圧の上限により、加速エネルギーの上限が決められてしまう。これを克服するために繰り返し加速電場(Radio Frequency:RF)を通過させるコンセプトが生まれた。

RF加速器の形状は線形加速器と円形加速器の2種類がある。円形加速器中の荷電粒子が一様磁場中を運動する場合、ローレンツ力と遠心力によるつり合いで運動方程式が記述される。粒子の速度が光速に比べて小さい非相対論的な場合、質量は静止質量に等しいので周波数は獲得するエネルギーによらずほぼ一定となる。粒子の速度が光速に比べて無視できない場合、等時性は失われるので加速周波数との同期をとる必要がある。同期の条件を満たした円形加速器はシンクロトロンと呼ばれ、軌道半径を一定に保ちながら荷電粒子を加速し、偏向電磁石の磁場の強度を加速にあわせて調整している。シンクロトロンの加速における問題は、粒子の周回周期と高周波の同期を如何に取るかという点にあった。この問題は1945年にロシアのヴェクスラーとアメリカのマクミランによって独立に位相安定性の原理\cite{phase_focusing}が発見され解決された。軌道半径が変わらないことにより、磁場は軌道部分だけにあればよくなるので、電磁石が小型化された。その後、1952年にクーラン、リビングストンらによって発見された強集束の原理\cite{Courant}によって、更に加速器全体の小型化が可能となった。これらの原理が加速器の高エネルギー化に貢献した。

高エネルギーで加速された相対論的速度で運動する電子ビームは、磁場によって曲げられたとき、その加速度方向に電磁波を発する。この現象はシンクロトロンで初めて観測され\cite{radiation}、当時はエネルギー損失となる障害と考えられていた。しかし、原理的にどのような波長帯域でも発生することが可能であり、鋭い指向性と高い放射パワーから光源サイズや発散角をレーザー並みに小さくできることから放射光源としての有用性が見出され、放射光源としての加速器が開発されてきた。

放射光源としての加速器である蓄積リング型放射光源を運転するためには、線形加速器もしくはブースターシンクロトロンで目的のエネルギーまで加速された後、ビーム輸送路を通して主リングを入射する構成が必要である。主リングを周回する電子ビームがシンクロトロン放射して放射光を発生する。加速器は以下のコンポーネントで構成される複雑なシステムである。システムの構成には電子ビームを曲げ円軌道にするための多数の偏向電磁石、電子ビームの軌道を集束させるための4極電磁石、粒子を加速させるためのRF空洞、強い輝度を得るために使われる挿入光源、入射に用いられるキッカー電磁石やセプタム電磁石など多くのコンポーネントが挙げられる。これら全てのコンポーネントを制御して、高輝度、高精度のビームを安全に安定して運用することが要求される。
%*******************************************************************************
\section{Photon Factory}
茨城県つくば市に高エネルギー加速器研究機構 (KEK)の蓄積リング型放射光源 (Photon Factory Ring:PF)がある\cite{PF}。PFは1982年の運転開始以来、2002年の高輝度化と2005年の直線部増強を経て、表\ref{Parameter_PF}に示す基本性能を有する。図\ref{Lattice_PF}と図\ref{Optics_PF}に、PFリングの電磁石の配置図とオプティックスを示す。
%*******************************************************************************
\section{LINAC}
KEK電子・陽電子入射器(LINAC)は全長600 mの線形加速器である\cite{Linac}。LINACは異なる5つのリングSuperKEKB HER, Positron Damiping Ring, SuperKEKB LER, PF, PF-ARにビームを入射しており、繰り返し50 \si{Hz}で任意のリングへ入射可能である。PFには2.5 GeVのフルエネルギーまで加速した後、ビームスイッチヤードで振り分けてビーム輸送路  (Beam Transport: BT)を通し直接入射している。図\ref{Lattice_BT}と図\ref{Optics_BT}に、BTの電磁石の配置図とオプティックスを示す。
%*******************************************************************************
\section{研究目的}
本論文はPFリングで実施した入射スタディをまとめたものである。スタディの主な目的は入射率の改善、つまりビーム輸送におけるロスを改善することにある。PFリングではキッカー電磁石に設定値から振幅や励磁タイミングの誤差を含むことやマルチターンキック入射の影響のため、入射時の蓄積ビームの振動や入射率低下の問題があった\cite{Kobayashi}。そこに2011年の東日本大震災によるビーム輸送路の電磁石やセプタム電磁石等のアライメント誤差などが原因で、入射率の低下が問題となっている。特に入射セプタム電磁石は真空槽内にありターゲット座がないため再配置されていない。そこで本研究は、直接ビームを測定するビームベースで入射パラメータの測定とそれらの最適化を実施した。
%*******************************************************************************
\newpage
\section{本論文の構成}
本論文の構成を説明する。

第2章では、ベータトロン振動に関する単粒子力学の理論を記述する。はじめに加速器電磁石の多極展開を説明する。次にHill's equationからベータトロン振動の導出について説明をする。最後にベータトロン振動の運動を位相平面上に写像する手法を説明する。

第3章では、入射方式について記述する。はじめに理想的な入射であるキッカー入射について説明する。つぎに、キッカー入射に含まれる誤差を説明し、問題としているマルチターンキック入射について説明する。

第4章では、入射パラメータの測定を記述する。ここで測定した入射パラメータは、入射点のビーム位置関係とキッカー、セプタム電磁石の励磁曲線である。

第5章では、入射シミュレーションを記述する。入射シミュレーションでは、キッカーバンプの計算と入射率の計算を行う。ここでは理想的なシングルターンキック入射を示した後、問題としているマルチターンキック入射が及ぼす影響を再現させるために行った内容についてまとめる。

第6章では、入射最適化を説明する。最適化方法で必要な評価関数とアルゴリズム、測定方法について述べる。最適化結果で最適化前後の振動の変化、入射シミュレーションで得られる最適解との違いをまとめた。

第7章では、研究の成果をまとめ、今後の展望を記述する。

\newpage
    \begin{figure}[H]
        \centering
        \includegraphics[width=130mm]{./figure/01_Introduction/lattice_ring.png}
        \caption{PFリングのラティス構造}
        \label{Lattice_PF}
    \end{figure}

    \begin{figure}[H]
        \centering
        \includegraphics[width=100mm]{./figure/01_Introduction/optics_ring.png}
        \caption{PFリングのオプティックス}
        \label{Optics_PF}
    \end{figure}

    \begin{table}[htbp]
    \caption{PFリングの主要パラメータ}
    \label{Parameter_PF}
    \centering
    \begin{tabular}{ l  l c } \hline
    General parameters & & \\
    Energy & E[GeV] & 2.5  \\
    Circumference & L[m] & 187.4074  \\
    Natural emittance & $\varepsilon_0$ & 35.4 \\
    Current & I[mA] & 450 \\
    Revolution time & T[ns] & 624 \\
    Energy spread & $\sigma$ & $7.29E^{-4}$ \\
    Momentum Compaction & $\alpha$ & $6.56E^{-3}$ \\
    Horizontal tune & $\nu_x$ & 9.6 \\
    Vertical tune & $\nu_y$ & 5.3 \\
    Horizontal chromaticity & $\chi_x$ & $-13.4$ \\
    Vertival chromaticity & $\chi_y$ & $-15.8$ \\
    Coupling factor & & 0.01 \\
    Horizontal damping time & $\xi_x$[msec] & $7.8$ \\
    Vertical damping time & $\xi_y$[msec] & $7.8$ \\
    Longitudinal damping time & $\xi_s$[msec] & $1.6$ \\
    RF parameters & & \\
    RF frequency & [MHz] & 500 \\
    Revolution frequency & [MHz] & 6.56 \\
    RF Voltage & $V_rf$[MV] & 1.7 \\
    Synchrotron tune & $\nu_s$ & -0.015 \\
    Bunch length & $\sigma_z$[mm] & 9.7 \\
    RF bucket & $\frac{\Delta E}{E}_{RF}$ & 1.180 \\ \hline
    \end{tabular}
    \end{table}
\newpage
    \begin{figure}[H]
        \centering
        \includegraphics[width=150mm]{./figure/01_Introduction/lattice_bt.png}
        \caption{輸送路のラティス構造}
        \label{Lattice_BT}
    \end{figure}

    \begin{figure}[H]
        \centering
        \includegraphics[width=130mm]{./figure/01_Introduction/optics_bt.png}
        \caption{輸送路のオプティックス}
        \label{Optics_BT}
    \end{figure}
%*******************************************************************************
\newpage
\chapter{単粒子力学}
円形加速器を周回する粒子の軌道は基準軌道とその周りの振動に分けられる。基準軌道周りの横方向の振動をベータトロン振動と呼ぶ。本章では、ベータトロン振動の運動方程式を導出し、粒子の軌跡を位相平面にプロットして、軌跡を考える方法について説明する。運動方程式の導出はOHOセミナーの資料\cite{Hochi}を参考にした。

加速器を周回するビームは、偏向電磁石で曲げられて円軌道に、4極電磁石で集束されてダクト内に閉じ込められる。光学レンズ凸レンズはレンズ全体で集束になるが、電磁石の場合、進行方向に向かって左右 (水平)が集束 (凸レンズ的)であれば上下 (垂直)は発散 (凹レンズ的)になる。従って加速器では基本的に凹レンズと凸レンズを並べて、縮めて拡げて、を繰り返して真空ダクト内を輸送することとなる。4極電磁石の力は保存力であり、中心軌道に沿った電子の運動は変調された調和振動となる。調和振動子の場合、位置と運動量とで張られる位相空間内の軌跡を描くと円になる。加速器の場合、加速器の場所場所で変調に応じて規格化した位相空間に対しては滑らかな円運動となる。そのような水平及び垂直方向の運動は独立であり、1周の周波数をベータトロン周波数 (チューンともいい、水平と垂直がある)という。加速器の1カ所でビームを観測した場合、規格化した位相空間では円上をチューンに応じて跳び跳びに電子が観測される。チューンが整数であれば常に同じ場所、半整数であれば2カ所を交互に行き来するが、一般的な加速器ではベータトロン振動の共鳴は振幅増大でビームロスにつながるため、共鳴を避けるためにチューンの端数をわざと半端な数にする。その為、長い時間観測すると飛び飛びの電子がだんだん円周を埋めていく、ということになる。そのような位相空間プロットをポアンカレプロットという。

加速器物理的には、電磁石からの力は保存力のため、位相空間の面積は保存 (リュービルの定理が成立)し、その保存量をクーランシュナイダー不変量という。入射ビームが軸から外れた部分に入射されると、以後、解析的には位相空間のある円上で、面積保存のもと、ベータトロン振動を行うことになる。現実には、非線形磁場や誤差磁場の影響で、少しずつずれが生じていくし、放射光を出す効果は保存力と全く別の効果を引き起こす。

粒子が放射光を出すと、粒子の進行方向のエネルギー (運動量)が失われる。粒子が設計軌道に対して傾いていれば、傾いた方向の運動量が減る。一方、エネルギー損失を埋め合わせるRFによる加速は常に中心軌道に沿った向きになる。従って、放射光を出し、RFで加速され、を繰り返すと、粒子の余計な方向の運動量が減ってゆき、ベータトロン振動が減衰していくことになる。それを放射減衰という。入射ビームの大振幅のベータトロン振動は、放射減衰によって減衰してゆき、やがて既存の蓄積ビームに混じり合うことになる。

%*******************************************************************************
\section{加速器電磁石要素の多極展開}
真空中のマクスウェル方程式は、
\begin{eqnarray}
\label{maxwell}
\bm{\nabla} \cdot \bm{D} = 0  \\
\bm{\nabla} \cdot \bm{B} = 0  \\
\bm{\nabla} \times \bm{E} + \frac{\partial \bm{B}}{\partial t} = 0 \\
\bm{\nabla} \times \bm{H} - \frac{\partial \bm{D}}{\partial t}= 0
\end{eqnarray}
で表記される。式(\ref{maxwell})の二次元の円筒座標による磁場の展開を考える。取り扱う空間内に電流がない場合、磁場はスカラーポテンシャル$\varphi$とすると、
\begin{eqnarray}
\bm{\nabla} \times \bm{B}  &=& 0 \\
\bm{B}      &=& \bm{\nabla} \varphi \\
\bm{\nabla} \cdot \bm{B} &=& \Delta \varphi = 0
\label{laplace}
\end{eqnarray}
となる。式(\ref{laplace})のラプラス方程式を変数分離$\varphi = R(r)\Theta(\theta)$を使って,
\begin{eqnarray}
\frac{d^2\Theta}{d\theta^2} &=& -m^2\Theta \\
\left(\frac{1}{r} \left(\frac{d}{dr} r \frac{d}{dr} \right) - \frac{m^2}{r^2} \right) R &=& 0
\end{eqnarray}
に変換する。前式は$\theta$に対して周期的な解を持ちmは整数となる。
\begin{equation}
\Theta = e^{-im\theta}, \theta \to \theta +2\pi
\end{equation}
後式は、
\begin{equation}
R = ar^n + br^{-n}
\end{equation}
になる。但し、$n=0$の時は正則でないので除かれる。原点$r=0$で正則であり、2次元円筒座標系における磁場は、
\begin{equation}
\varphi = \sum_{n=1}r^n(A_n \sin n\theta +B_n\cos n\theta)
\end{equation}
となる。$A_n$がnormal成分、$B_n$がskew成分という。$\bm{B} = \bm{\nabla} \varphi$より、
\begin{eqnarray}
B_r &=& \frac{\partial \varphi}{\partial r} = \sum_{n=1}nr^{n-1}(A_n\sin n\theta +B_n \cos n\theta) \\
B_\theta &=& \frac{1}{r} \frac{\partial \varphi}{\partial \theta} = \sum_{n=1}nr^{n-1}(A_n \cos n\theta - B_n\sin n\theta)
\end{eqnarray}
となる。これを直交座標の多極展開磁場になおして、$n=1$の時の2極成分は、
\begin{eqnarray}
B_x &=& B_1\\
B_y &=& A_1
\end{eqnarray}
$n=2$の時の4極成分は、
\begin{eqnarray}
B_x &=& 2A_2y + 2B_2x \\
B_y &=& 2A_2x + 2B_2y
\end{eqnarray}
高次成分は、
\begin{eqnarray}
A_n &=& \frac{1}{n!} \left( \frac{\partial^{n-1}}{\partial r^{n-1}}B\theta \right) \bigg|_{\theta=0} \\
B_n &=& -\frac{1}{n!} \left( \frac{\partial^{n-1}}{\partial r^{n-1}}B\theta \right) \bigg|_{\theta=\frac{\pi}{n}}
\end{eqnarray}
と記述される。
    \begin{figure}[H]
        \centering
        \begin{tabular}{c}
        \begin{minipage}{0.45\hsize}
        \centering
        \includegraphics[clip, width=0.9\columnwidth]{./figure/02_Theory_Dynamics/dipole_magnet.png}
        \hspace{16mm} [1]Dipole Magnet (n=1)
        \end{minipage}
        \begin{minipage}{0.45\hsize}
        \centering
        \includegraphics[clip, width=0.9\columnwidth]{./figure/02_Theory_Dynamics/quadrupole_magnet.png}
        \hspace{16mm} [2]Quadrupole Magnet (n=2)
        \end{minipage}
        \end{tabular}
        \caption{2極、4極成分に対応する磁極配置}
        \label{Componet_AC}
    \end{figure}
%*******************************************************************************
\newpage
\section{Hill's equation}
座標は直交曲線座標系を用いる。設計軌道が水平面にある場合で水平面内の外向き方向、鉛直方向、軌道接線方向を$(x,y,s)$で記述する(図\ref{Coordinate})。
    \begin{figure}[H]
        \centering
        \includegraphics[width=110mm]{./figure/02_Theory_Dynamics/orthogonal_curvilinear.png}
        \caption{直交曲線座標系}
        \label{Coordinate}
    \end{figure}
電子の位置は設計軌道の周りに変位を加えて、
\begin{equation}
r(s) = \rho(s) + x\bm{e_x} + y\bm{e_y}
\end{equation}
で記述され、$\rho(s)$は設計軌道(Reference orbit)の曲率半径を表す。加速器では、磁場の分布が与えられると粒子のエネルギーに応じて、一定の閉じた軌道がLorentz力と遠心力のつり合いから求まる。この設計軌道の周りでベータトロン振動と呼ばれる横方向の振動が生じる。この振動の運動方程式を導く。電磁場中の運動をする荷電粒子のハミルトニアンは、
\begin{equation}
H_1 = e\phi + c \left[ m^2c^2+(\bm{P}-e\bm{A})^2 \right]^2
\end{equation}
で与えられる。ここで、eは電荷量、cは光速、mは粒子の静止質量、$\phi$はスカラーポテンシャル、$\bm{A}$はベクトルポテンシャルを表す。ベクトルポテンシャル$\bm{A}$とスカラーポテンシャル$\phi$、電磁場は、
\begin{eqnarray}
\bm{E} &=& -\nabla \phi-\frac{\partial \bm{A}}{\partial t} \\
\bm{B} &=& \nabla \times \bm{A}
\end{eqnarray}
の関係を持つ。また、一般運動量は、
\begin{equation}
\bm{P} = \bm{p} + e\bm{A}
\end{equation}
で与えられる。直交座標化から直交曲線座標系(図\ref{Coordinate})への座標変換では、sは設計軌道に沿った距離で時間tの代わりに使用される。直交曲線座標系での単位ベクトルは、
\begin{eqnarray}
\bm{e}_s(s) &=& \frac{d\bm{r}_0}{ds}                \nonumber \\
\bm{e}_x(s) &=& -\rho(s)\frac{d\bm{e}_s(S)}{ds} \nonumber \\
\bm{e}_y(s) &=& -\bm{e}_s(s)\times \bm{e}_x(s)
\end{eqnarray}
と記述される。次の母関数、
\begin{equation}
F_3(\bm{P},x,y,s) = -\bm{P} \cdot (\bm{r}_0 + x\bm{e}(s) + y\bm{e}_y(s))
\end{equation}
を用いて正準変換を行うと、一般運動量は、
\begin{eqnarray}
p_s &=& - \frac{ \partial F_3}{ \partial s} = \left(1+\frac{x}{ \rho} \right) \bm{P} \cdot \bm{e}_s(s) \nonumber \\
p_x &=& -\frac{ \partial F_3}{\partial x} = \bm{P} \cdot \bm{e}_x(s) \nonumber \\
p_y &=& -\frac{ \partial F_3}{ \partial y} = \bm{P} \cdot \bm{e}_y(s)
\end{eqnarray}
となり、同様にベクトルポテンシャルも、
\begin{eqnarray}
A_s &=& \left(1+\frac{x}{ \rho} \right) \bm{A} \cdot \bm{e}_s(s) \nonumber \\
A_x &=& \bm{A} \cdot \bm{e}_x(s) \nonumber \\
A_y &=& \bm{A} \cdot \bm{e}_y(s)
\end{eqnarray}
と変換して、ハミルトニアンは、
\begin{eqnarray}
\lefteqn{ H_2(x,p_x,y,p_y,s,p_t) } \nonumber \\
&=&  H_1 + \frac{\partial F_3}{\partial t} \nonumber \\
&=&  e\phi +c\left[m^2c^2+(p_x-eA_x)+(p_y-eA_y)^2+\frac{(p_s-eA_s)^2}{(1+x/p)^2} \right]^{1/2}
\label{Hamiltonin_1}
\end{eqnarray}
となる。次に、距離sを新たな独立変数として取り扱う。新たなハミルトニアンは、式(\ref{Hamiltonin_1})を$-p_s$について解くことにより、
\begin{eqnarray}
\lefteqn{ H_3(x,p_x,y,p_y,t,-H_2:s) } \nonumber \\
&=&  -p_s \nonumber \\
&=&  -eA_s - \left(1+\frac{x}{\rho} \right)\left[ \frac{(H_2-e\phi)^2}{c^2}-m^2c^2-(p_x-eA_x)^2-(p_y-eA_y)^2 \right]^{1/2}
\label{Hamiltonin_2}
\end{eqnarray}
で与えられる。ここで、電場はなく、磁場は静的で、設計軌道に対して平行な磁場成分を持たないという条件、
\begin{equation}
\phi = 0 , A_x = A_y = 0
\label{jouken}
\end{equation}
を仮定する。この場合、式(\ref{Hamiltonin_2})は、時刻tを含まないので全エネルギーは不変量となる。$p_x$、$p_y$について2次の項まで考慮すると、
\begin{eqnarray}
\lefteqn{ H_3(x,p_x,y,p_y) } \nonumber \\
&=& -eA_s-p\left(1+\frac{x}{ \rho}\right)\left[1-\left(\frac{p_x}{p}\right)^2-\left(\frac{p_y}{p}\right)^2 \right]^{1/2}   \nonumber \\
&\simeq& -eA_s - p \left(1+\frac{x}{ \rho} \right) + p \left(1+\frac{x}{ \rho} \right)\left[\frac{1}{2} \left(\frac{p_x}{p} \right)^2-\frac{1}{2}\left(\frac{p_y}{p} \right)^2 \right]
\end{eqnarray}
となる。上記の独立変数sでのハミルトニアン方程式は、
\begin{eqnarray}
x'   &=& \frac{dx}{ds} = \frac{\partial H_3}{\partial p_x}     \\
p'_x &=& \frac{dp_x}{ds} = -\frac{\partial H_3}{\partial x}   \\
y'   &=& \frac{dy}{ds} = \frac{\partial H_3}{\partial p_y}     \\
p'_y &=& \frac{dp_y}{ds} = -\frac{\partial H_3}{\partial y}
\end{eqnarray}
と与えられる。直交曲線座標系での磁場成分は、
\begin{eqnarray}
\bm{B} &=& \frac{1}{ \left(1+\frac{x}{ \rho} \right)}\left( \frac{\partial A_s}{\partial y}- \frac{\partial A_y}{\partial s} \right) \bm{e}_x \nonumber \\
&+& \frac{1}{ \left(1+\frac{x}{ \rho} \right)}\left( \frac{\partial A_x}{\partial s}- \frac{\partial A_s}{\partial x} \right) \bm{e}_y \nonumber \\
&+& \left( \frac{\partial A_y}{\partial x}- \frac{\partial A_x}{\partial y} \right) \bm{e}_s 
\end{eqnarray}
となる。上式と式(\ref{jouken})を整理して、磁場の$x$、$y$成分は、
\begin{eqnarray}
B_x &=& \frac{1}{1+x/\rho}\frac{\partial A_s}{\partial y} \nonumber \\
B_y &=& \frac{1}{1+x/\rho}\frac{\partial A_s}{\partial x}
\label{B_xy}
\end{eqnarray}
となり、式(\ref{Hamiltonin_2})、式(\ref{jouken})、式(\ref{B_xy})から、$x$、$y$について2次の項までを考慮すると、ベータトロン振動の運動方程式は以下のように求まる。
\begin{eqnarray}
x'' - \frac{\rho+x}{\rho^2} = - \frac{B_y}{B\rho} \frac{p_0}{p} \left( 1+\frac{x}{\rho} \right)^2 \nonumber \\
y'' = \frac{B}{B\rho} \frac{p_0}{p} \left( 1+\frac{y}{\rho} \right)^2
\label{eq_hills_be}
\end{eqnarray}
式(\ref{eq_hills_be})の$p_0$は基準軌道上を周回する粒子の設計運動量、Bはローレンツ力のつり合いを満たす偏向磁場を指す。前節の2極磁場と4極磁場の加速器要素を代入した線形のベータトロン振動の運動方程式は
\begin{eqnarray}
x'' + \left( \frac{1}{\rho^2}+\frac{1}{B\rho} \frac{\partial B_y}{\partial x} \right) x &=& 0  \nonumber \\
y'' - \frac{1}{B\rho} \frac{\partial B_y}{\partial x} &=& 0
\label{eq_hills_af}
\end{eqnarray}
と得られる。
%*******************************************************************************
\section{ベータトロン振動}
式(\ref{eq_hills_af})で導かれた線形のHill's equationをsに依存したK(s)を使うことで以下のようにまとめる。
\begin{eqnarray}
\frac{d^2 x}{d s^2} + K(s)x &=& 0 \nonumber \\
\frac{d^2 y}{d s^2} + K(s)y &=& 0
\label{eq_hills}
\end{eqnarray}
係数K(s)は円形加速器の場合、磁石の並びの周期性(ラティス構造)に従って$K(s)=K(s+L)$の周期関数となる。周期条件を満たす構造の振動は、フロケーの定理より振幅関数と位相に分解することができ、方程式の一般解は、
\begin{equation}
\chi = c_1w(s)e^{+i\phi(s)}+c_2w(s)e^{-i\phi(s)}
\end{equation}
と書ける。式(\ref{eq_hills})に一般解を代入し、方程式が任意の初期位相について成立するためには、
\begin{eqnarray}
w'' + K(s)w - 1/w^3 &=& 0 \\
\phi' &=& 1/w^2
\label{beta_equation}
\end{eqnarray}
を満たす必要があり、Twissパラメータを、
\begin{eqnarray}
\beta(s) &=& w^2(s) \nonumber \\ 
\alpha(s) &=& -w(s)w'(s) = -\beta'(s)/2 \\
\gamma(s) &=& \frac{1+w^2(s)w'^2(s)}{w^2(s)}
\end{eqnarray}
と定義する。この定義には、お互いに、
\begin{eqnarray}
\alpha(s) &=& -\frac{\beta''(s)}{2} \\
\gamma(s) &=& \frac{1+\alpha^2(s)}{\beta(s)}
\end{eqnarray}
の関係がある。式(\ref{beta_equation})は積分して、Twissパラメータを用いて方程式を書き直すと、
\begin{equation}
\varphi(s) = \int_0^s \frac{ds}{\beta(s)}
\end{equation}
で与えられる。これは、ベータトロン振動の位相進みを表し、リング一周の積分をして$2\pi$で割った値はチューンと呼び、
\begin{equation}
\nu = \frac{1}{2\pi}\oint \frac{ds}{\beta(s)}
\end{equation}
で与えられ、加速器一周当たりの振動数を表す。
線形微分方程式に従う位相空間での座標の移動は転送行列で表現でき、$s_1$と$s_2$を結ぶ転送行列$M(s_2|s_1)$は、
\begin{equation}
\left(
    \begin{array}{c}
    x(s_2) \\
    p_x(s_2)
    \end{array}
\right)
= M(s_2|s_1)
\left(
    \begin{array}{c}
    x(s_1) \\
    p_x(s_1)
    \end{array}
\right)
\end{equation}
\begin{eqnarray}
m_{11} &=& \sqrt{\frac{\beta_{s_2}}{\beta_{s_1}}}(\cos \phi_{s_2s_1}+\alpha_{s_2}\sin \phi_{s_2s_1}) \\
m_{12} &=& \sqrt{\beta_{s_2}\beta_{s_1}} \sin \phi_{s_2s_1} \\
m_{21} &=& \frac{\alpha_{s_2}-\alpha_{s_1}}{\sqrt{\beta_{s_2}\beta_{s_1}}} \cos \phi_{s_2s_1} - \frac{1+\alpha_{s_2}\alpha_{s_1}}{\sqrt{\beta_{s_2}\beta_{s_1}}} \sin \phi_{s_2s_1} \\
m_{22} &=& \sqrt{\frac{\beta_{s_2}}{\beta_{s_1}}}(\cos \phi_{s_2s_1}+\alpha_{s_2}\sin \phi_{s_2s_1}) \\
\phi_{s_2s_1} &=& \phi_{s_1}-\phi_{s_2}
\end{eqnarray}
で記述される。一周の転送行列は、
\begin{equation}
M_C =
\left(
    \begin{array}{ccc}
      \cos \mu + \alpha \sin \mu & \beta \sin \mu \\
      -\gamma \sin \mu & \cos \mu - \alpha \sin \mu
    \end{array}
\right)
\end{equation}
と書ける。エンヴェロープの方程式は、
\begin{equation}
\sqrt{\beta}''+K(s)\sqrt{\beta}-\frac{1}{\sqrt{\beta}^3} = 0
\end{equation}
となる。ベータトロン振動の一般解は、
\begin{eqnarray}
\label{eq_betatron_oscillation}
x(s) &=& \sqrt{\beta(s)J}\cos \left(\int_0^s\frac{ds}{\beta(s)}+\varphi_0 \right) \\ \nonumber
x'(s) &=& -\sqrt{ \frac{2J}{\beta(s)} } \left[ \alpha(s)\cos (\varphi(s)+\varphi_0)+\sin (\varphi(s)+\varphi_0) \right]
\end{eqnarray}
と書ける。ここで$\varphi_0$は初期条件により決定される定数である。Jは次の不変量を表す。
\begin{equation}
J = \frac{1}{2\pi}\int_{torus}ds'ds = \frac{1}{2\pi}\oint s'ds
\end{equation}
ベータトロン関数$\beta(s)$は、加速器の磁場分布によって特徴づけられ振幅に関係する。
%*******************************************************************************
\newpage
\section{ポアンカレプロット}
ベータトロン振動の式(\ref{eq_betatron_oscillation})から、
\begin{equation}
\beta x' + \alpha x = -\alpha \beta^{1/2}(s) \sin (\nu \varphi(s) + \delta)
\end{equation}
の関係が得られる。不変量を次の式でまとめ、
\begin{equation}
\gamma x^2 + 2\alpha xx' + \beta x'^2 = 2J = const.
\end{equation}
上式をクーランシュナイダー不変量(Courant-Synder invariant)と呼ぶ\cite{Courant_inv}。この式は、円形加速器において運動する粒子は位相平面上で楕円上を周回すること、Twissパラメータの作用を表すことを示している。図\ref{Phase_space}の左図に1ターン目の$(x, p_x)$から出発した粒子が位相平面上で1周後に通る点をプロットした、ポアンカレプロットを示す。楕円の線はチューンの小数が適当な値の時、粒子が通る軌跡の点をつなげた線に相当する。また、この楕円は次の式、
\begin{eqnarray}
X &=& \frac{x}{\sqrt{\beta}} \\
P &=& \frac{\alpha x + \beta x'}{ \sqrt{\beta}}
\end{eqnarray}
で規格化される。これを図\ref{Phase_space}の右図に示す。
    \begin{figure}[H]
        \centering
        \begin{tabular}{c}
        \begin{minipage}{0.45\hsize}
        \centering
        \includegraphics[clip, width=1.0\columnwidth]{./figure/02_Theory_Dynamics/Phase_space.png}
        \hspace{16mm} [1]位相空間
        \end{minipage}
        % 2
        \begin{minipage}{0.45\hsize}
        \centering
        \includegraphics[clip, width=1.0\columnwidth]{./figure/02_Theory_Dynamics/Normalized_phase_space.png}
        \hspace{16mm} [2]規格化位相空間
        \end{minipage}
        \end{tabular}
        \caption{単一粒子の位相空間上の軌跡}
        \label{Phase_space}
    \end{figure}

%*******************************************************************************
\chapter{入射方式}
本章では、PFで採用しているキッカー入射とキッカー入射に関連する入射方式について説明する。蓄積リングを周回する電子ビームは、ガス散乱や量子寿命、ビーム内散乱によって時間と共に減ってゆく\cite{Wrulich}。これを回復するため入射器から供給されるビームを、線形加速加速器もしくはブースターシンクロトロンで加速して、輸送路を通しリングへ注ぎ足す。この輸送路からリングへの受け渡しを入射と呼ぶ。入射では入射率を最大にすることが重要である。
%*******************************************************************************
\section{パルス電磁石による入射}
入射は位相平面を使って記述される。位置と角度が$(x, p_x)$で入射されたビームのクーランシュナイダー不変量は、
\begin{equation}
2J = \gamma x^2 + 2\alpha xx' + \beta x'^2
\end{equation}
であった。規格化は次の式、
\begin{eqnarray}
X &=& \frac{x}{\sqrt{\beta}} \\
P &=& \frac{\alpha x + \beta x'}{ \sqrt{\beta}}
\end{eqnarray}
で行われる。規格化座標で図\ref{Off-axis}の入射を考える。
     \begin{figure}[H]
        \centering
        \includegraphics[width=80mm]{./figure/03_Theory_Injection/accep_exam.png}
        \caption{アクセプタンスと入射ビームの振舞い}
        \label{Off-axis}
    \end{figure}
図\ref{Off-axis}のアクセプタンスはセプタム壁まで距離を表す。入射では初期振幅をセプタム壁より内側に収めることが必要である。入射ビームの初期振幅は規格化座標を用いて、
\begin{equation}
A_{inj}^2 = X_0^2 + P_0^2
\end{equation}
で表される。このとき初期振幅$A_{inj}^2$は図\ref{Off-axis}で、実線の半径に相当する。図\ref{Off-axis2}に、入射されたビームがリングを一周してセプタム壁にあたり失われる様子を示す。
     \begin{figure}[H]
        \centering
        \includegraphics[width=120mm]{./figure/03_Theory_Injection/accep_exam2.png}
        \caption{入射ビームがセプタム壁にあたって失われる様子}
        \label{Off-axis2}
    \end{figure}
アクセプタンス内に収めるだけであればパルス電磁石のダイポールで振幅を減らすだけでよい。ダイポールマグネットで角度$\theta$蹴り込み、振動を抑制させた様子を図\ref{Off-axis3}と図\ref{Off-axis4}にプロットした。
     \begin{figure}[H]
        \centering
        \includegraphics[width=120mm]{./figure/03_Theory_Injection/accep_exam3.png}
        \caption{入射ビームがダイポールマグネットで蹴られた後の入射振動}
        \label{Off-axis3}
    \end{figure}
     \begin{figure}[H]
        \centering
        \includegraphics[width=80mm]{./figure/03_Theory_Injection/accep_exam4.png}
        \caption{入射ビームがダイポールマグネットで蹴られた後の入射振動(位相空間)}
        \label{Off-axis4}
    \end{figure}
蹴り込まれたビームはリング内に入射され、内側の実円の上を回転しながら、放射減衰して定常状態のビームの大きさになる。既に蓄積されたビームが存在する場合、ダイポールキックは基準軌道を通っていた蓄積ビームの振動を起こす(図\ref{Off-axis5})。
     \begin{figure}[H]
        \centering
        \includegraphics[width=120mm]{./figure/03_Theory_Injection/accep_exam5.png}
        \caption{入射ビームと蓄積ビームがダイポールマグネットで蹴られた後の入射振動}
        \label{Off-axis5}
    \end{figure}
蓄積ビームに影響を与えることなく、入射するため、通常の放射光施設では次節のキッカー入射が行われる。
%*******************************************************************************
\section{キッカー入射}
     \begin{figure}[H]
        \centering
        \includegraphics[width=120mm]{./figure/03_Theory_Injection/scheme_kicker_inj.png}
        \caption{入射システムの概念図}
        \label{tv_kicker_injection}
    \end{figure}

蓄積ビームに影響を与えることなく入射ビームの振幅を小さくする為に、蓄積リングでは一般的に図\ref{tv_kicker_injection}に示すバンプ軌道を用いた入射方式が用いられる\cite{kicker_inj}。この入射方式では、入射の瞬間にキッカー電磁石とセプタム電磁石の2種類のパルス電磁石を励磁する。セプタム電磁石はリングの入射点直前に設置されている電磁石であり、入射ビームの軌道の向きを蓄積リングの設計軌道と平行の向きにするために用いられる。セプタムでは入射ビームのみを曲げ、蓄積ビームには影響を与えないようにする必要がある。そこで、入射ビームと蓄積ビームとの間をセプタム板という銅板で区切り、電磁石をパルス的に励磁することによる渦電流の効果を利用して磁場を遮蔽している。キッカー電磁石は、入射点に局所バンプを形成する様に配置される。通常、4台のキッカー電磁石が、入射点上流に2台、下流に2台配置される。入射ビームの振幅を、局所バンプによって小さくすることができれば、入射ビームは入射点に戻ってもセプタム壁に衝突することなく、周回し続けることが出来る。アパーチャー内に収められた入射ビームは放射減衰の効果により、数万周回後(数ミリ秒)には蓄積ビームに混ざり合うことになる。

図\ref{Phase_space_kicker_inj}に位相空間での入射ビームと蓄積ビームの軌道とセプタム電磁石の壁の関係を示す。

    \begin{figure}[H]
        \centering
        \includegraphics[width=100mm]{./figure/03_Theory_Injection/Four_kicker.png}
        \caption{4台のキッカー入射における入射ビームと蓄積ビームの軌跡}
        \label{Phase_space_kicker_inj}
    \end{figure}

図\ref{Phase_space_kicker_inj}でも蓄積ビームは基準軌道に戻され、入射振幅はセプタム内壁までの距離より抑制されていることがわかる。他の部分でアクセプタンスを制限するときは、その部分に合わせた振幅まで小さくする。本論ではアクセプタンスを制限する可能性のあるコンポーネントとして、放射光から保護するために挿入されるアブソーバを考えた(表\ref{Aperture_phy})。

%*******************************************************************************
\section{キッカー入射の誤差}
バンプ軌道が閉じない場合、蓄積ビームにベータトロン振動が励起され、放射光強度に変動が生じ放射光実験に影響を与える。そのため全てのキッカー電磁石でパルス波形が一致し、励磁タイミングが同期されることが要求される。しかし、電磁石の製作精度や材料による個体差、ケーブル長の違いによるタイミング差、チェンバーコーティングの均一性など技術的な困難が含まれる。また電磁石の蹴り角の制御では、リングの設置前に磁場と電流特性の線形性を測定して制御に使用しているため、設置後の環境変化が含まれない。PFリングで使用している電磁石のパルス幅(Zero-to-Zero)で周回時間以内で閉じるべきが、それより長いため入射ビームに対してマルチターンにわたり蹴っている。またバンプ軌道の中にある6極電磁石が、エネルギーと振幅の2乗に依存して蹴り角を与えるので、設定のタイミング以外に対して完全なバンプとならない。ここでは、バンプマッチングのエラーとして、キッカー電磁石のパルス波形の誤差を以下の要素で考えた。図\ref{PS_exam}にその誤差を含んだパルス波形を示す。

     \begin{figure}[H]
        \centering
        \includegraphics[width=100mm]{./figure/03_Theory_Injection/PulseShape_Ideal.png}
        \caption{パルス波形の誤差}
        \label{PS_exam}
    \end{figure}

\begin{itemize}
\item  蹴り角

設定値から振幅や励磁タイミングに含まれる誤差の原因として制御系の誤差が考えられる。蹴り角の制御はリング設置前に磁場測定を行い、測定した励磁曲線を用いる。これは実験室測定によって決定したパラメータである。しかしビームベースアライメントしないと、十分な精度とならない場合がある\cite{injection_nsls}。またパルスの立ち上がり時間や立ち下がり時間、波形の不揃いによるミスマッチや渦電流、振幅のZero-to-Zero以降に含まれる反射などが蹴り角の誤差に含まれる。ここでミスマッチは設定値の蹴り角の比率からずれた値を表す。

\item  時間

リングの周回時間の2倍よりパルス幅が長いと入射ビームをマルチターンにわたってキックする。また、キッカーにパルス幅が異なる電磁石が個体差として含まれている場合、蹴り角やタイミングの調整でバンプを閉じることは原理的にできない。
\end{itemize}
%*******************************************************************************
\section{マルチターンキック入射}
この節では、理想ではシングルターンキックであるべきキッカーのパルス幅が周回時間の2倍より長いため、誤差として生じるマルチターンキックについて説明する。マルチターンキックの場合は、ターンごとのキックの振動を追いかけていくことで効果を検証することができる。図\ref{accep_mul}に、入射ビームがK3、K4によって蹴られアクセプタンスが小さく収められた振動が、2ターン目のKK1、KK2による蹴りでアクセプタンスが大きくなる様子を示す。また、ベータトロン振動は1次元線形であれば転送行列によって位相空間の変化は追跡できる。

     \begin{figure}[H]
        \centering
        \includegraphics[width=80mm]{./figure/03_Theory_Injection/accep_mul.png}
        \caption{マルチターンキックによる入射ビームの軌跡}
        \label{accep_mul}
    \end{figure}

マルチターンキックの影響は、蓄積ビームの振動の励起や初期振幅の増大など悪影響を与えることが予想される。また、実際のマルチターンキックではパルス波形の立ち上がり、ピーク、立下り、反射、それ以降の影響が含まれる。従って、調整やシミュレーションには全ての効果を含んだバンプ波形の再現が必要である。


%*******************************************************************************
\chapter{入射パラメータの測定}
本章では、入射パラメータの測定に関する内容を記述する。測定する入射パラメータは入射点でのビームの位置と傾きとキッカーのパルス波形、セプタムの励磁特性である。セプタムは十分にパルス幅が長いので幅や波形は問題にならない。現在の入射パラメータを調べ、入射シミュレーションのデザインパラメータと比較し、入射率低下の原因の特定と補正を行った。\ref{meas_method_inj}節でビームベースの測定方法について説明する、\ref{meas_result_inj}節で測定結果について記述する\cite{beam_based_meas}\cite{beam_based_meas2}。
\section{測定方法\label{meas_method_inj}}
入射パラメータの測定はビームベースで行う。ビームベース測定は、直線部 (ID02)両端のダクトの電極に検出回路 (Libera Brilliance+\cite{Libera})を接続してドリフトスペースを挟んだ2点の電子ビームの位置情報から、直線部中心の位相空間情報 (位置と傾き)を計算し、測定したリングのオプティックスで上流に転送する方法で行った(図\ref{sche_meas})。この測定で入射ビームの位相空間情報及びキッカー電磁石のパルス波形、セプタム電磁石・ビーム輸送路の補正電磁石の応答など入射パラメータを調べる。直線部からキッカー電磁石及び入射点を含めたリングの構造を図\ref{Lattice_meas}に示す。また、転送の計算で用いたTwissパラメータと直線部からの位相進みを表\ref{Twiss_ID02}にまとめる。

     \begin{figure}[htbp]
        \centering
        \includegraphics[width=130mm]{./figure/04_Measurement/sche_meas.png}
        \caption{測定の概略図}
        \label{sche_meas}
    \end{figure}

     \begin{figure}[htbp]
        \centering
        \includegraphics[width=130mm]{./figure/04_Measurement/Lattice_meas.png}
        \caption{測定の構成図}
        \label{Lattice_meas}
    \end{figure}

    \begin{table}[htbp]
    \caption{Twissパラメータと位相進み}
    \label{Twiss_ID02}
    \centering
    \begin{tabular}{ c c c c c c c } \hline
    Twiss parameter & K1 & K2 & K3 & K4 & ID02 & Inj \\ \hline
    $\alpha_x [rad]$ & -1.79 & -0.62 & -1.47 & -0.66 & 0.01 & -1.08 \\ 
    $\beta_x [\si{m}]$ & 3.25 & 6.02 & 13.8 & 3.15 & 12.0 & 9.45 \\ 
    $\Delta \phi_x$ & 8.01 & 8.29 & 1.24 & 0.86 & 0.00 & 1.26 \\ \hline
    \end{tabular}
    \end{table}

%*******************************************************************************
\newpage
\subsection{入射点でのビーム位置関係}

図\ref{CS_inj}にPFリングの蓄積ビームの中心軌道とバンプ、入射ビーム、そしてセプタム壁の位置関係を示す。図\ref{CS_inj}のパラメータは測量とキッカーの調整から推定した値である。これはセプタム電磁石の上流に設置されているアブソーバーの位置を測量して、アブソーバーの位置はダクト中心から15 \si{mm}と調べて、つぎにビームが削れるまで上流のキッカーの蹴り角を強くして、削れた時の蹴り角からセプタム内壁までの距離は21 \si{mm}だと推定している。

     \begin{figure}[htbp]
        \centering
        \includegraphics[width=100mm]{./figure/04_Measurement/CSinj.png}
        \caption{入射点での入射ビームと蓄積ビームの位置関係}
        \label{CS_inj}
    \end{figure}

入射には、ビームをアクセプタンス内に打ち込むことが要求され、調整には正確な位置情報が必要となる。しかしビーム輸送路の終端部では、ビームをBPMやスクリーンモニターで直接観測して調整を行うことはできない。それにモニター自身がずれてしまって、絶対値を失っている。そこで以下の方法から入射点で蓄積リングの軌道を基準としてのビームの位置関係を調べる。測定は、蓄積ビームの位置を測定して中心軌道を求めた後、入射ビームの座標を調べて相対座標に直した。

蓄積ビームの中心軌道の座標は、直線部2番両端のダクトの電極にビーム位置検出回路を接続して求める。ダクトには4つの電極があり、その信号の比U、Vから位置を導出できる。ビーム位置は、
\begin{equation}
\label{eq_beam_position}
x = \sum_{i=0}^3 \sum_{j=0}^3 k_x(i,j)U^iV^j 
\end{equation}
で表される。式(\ref{eq_beam_position})は、信号の比U, Vに非線形応答係数$k_x$で多項式補正した水平位置を示す。非線形応答係数はダクトの中心から離れた位置に対する出力電圧の非線形成分を補正する。直線部中心の位相空間情報は、式(\ref{eq_beam_position})を用いて、
\begin{eqnarray}
\label{trans_pos_PS}
x  =\frac{x_2+x_1}{2} \nonumber \\
x' =\frac{x_2-x_1}{l}
\end{eqnarray}
で表される。ここで、$l$(= 8.83  \si{m})は直線部の長さである。式(\ref{trans_pos_PS})から求めた水平方向の位相空間情報は、入射点の入射ビームやキッカー電磁石などの位相空間情報を求めるため、転送行列の逆行列で転送する。これは以下の式で示される。
\begin{equation}
\begin{pmatrix}
x \\
x'
\end{pmatrix}_{inj}
=
M(s_{ID02}|s_{inj})^{-1}
\begin{pmatrix}
x \\
x'
\end{pmatrix}_{ID02}
\end{equation}
転送は表\ref{Twiss_ID02}のtwissパラメータと位相進みから計算する。蓄積ビームの中心軌道および入射ビームの位相空間情報は、以上の方法で計算される。垂直方向に対しても同様である。
%*******************************************************************************
\subsection{キッカー電磁石}
キッカー電磁石のビーム応答測定では、蓄積ビームの振動を調べて、パルス波形を再構築しタイミングや蹴り角を調整した。キッカー電磁石のパルス波形は、以下の式を用いて、BPMで測定した直線部の位相空間情報を各キッカー電磁石の場所まで転送して蹴り角を導出し、パルスのタイミングを遅延時間方向に掃引した結果を繋げて求める。PFのキッカー電磁石のパルス時間長は周回時間より長いため、パルス波形を得るためにターンごとのキックの振動を追いかけて求めた。1ターン目のキックは直接観測され、パルス電磁石の蹴り角を$\theta(t)$、Kickerから検出回路を接続したBPMまでの転送行列を$M_{BK}$とおくと、
\begin{equation}
\label{meas_method_kicker}
\begin{pmatrix}
x_1 \\
x'_1
\end{pmatrix}_{BPM}
=
M(x_{BPM}|x_{Kicker})
\begin{pmatrix}
0 \\
\theta(t)
\end{pmatrix}_{Kicker}
\end{equation}
で表される。前ターンの蹴り角を含む振動は、
\begin{equation}
\begin{pmatrix}
x_{n+1} \\
x'_{n+1}
\end{pmatrix}_{BPM}
=
M(x_{BPM}|x_{Kicker}) \left \{
\begin{pmatrix}
x \\
x'
\end{pmatrix}_{Kicker}
+
\begin{pmatrix}
0 \\
\theta(t)
\end{pmatrix}_{Kicker}
\right \}
\end{equation}
が観測される。一方で、蹴り角が収束した応答は一周の転送行列を重ねて、
\begin{equation}
\begin{pmatrix}
x_{n+1} \\
x'_{n+1}
\end{pmatrix}_{BPM}
=
M(x_{BPM}|x_{Kicker})
\begin{pmatrix}
x_n \\
x'_n
\end{pmatrix}_{Kicker}
\end{equation}
となる。この計算を適用して、一周の転送との差を解き、ビームの応答からパルス波形を再構築した。

つまりパルス電磁石の蹴り角は、1ターン目は
\begin{equation}
x'_1 = \frac{1}{\sqrt{\beta_{Kicker}\beta_{BPM}}\sin \phi  } x_1
\end{equation}
で求められて、これを原点に2ターン目以降は、
\begin{equation}
\theta (t) = \frac{x_{n+1}-m_{11}x_n}{m_{12}}-x'_n
\end{equation}
で求められる。測定では遅延時間tを掃引して結果をプロットした。また、ビームベース測定によって蹴り角の制御を直接ビームでみた蹴り角に変換することができる。キッカー電磁石のパラメータを表\ref{PF_kicker}にまとめる\cite{PF_kicker}。

    \begin{table}[H]
    \caption{キッカー電磁石の設計値}
    \label{PF_kicker}
    \centering
    \begin{tabular}{ l c } \hline
    $\bm{Basic Parameters}$ &  \\
    Pulse length & 1.3 $\mu$sec \\
    Maximum kicker angle at 2.5 GeV & 4 mrad \\
    Total magnet length & 400 mm \\
    Gap height & 60 mm \\
    $\bm{Power Supply}$ & \\
    Impedance & 6.25$\Omega$ \\
    Charging voltage of PFL & 30 kV \\ \hline
    \end{tabular}
    \end{table}

%*******************************************************************************
\subsection{セプタム電磁石}

PFでは入射ビームの調整をビーム輸送路の上流の補正電磁石とセプタム電磁石S1、S2で行う。入射ビームのみを蹴ればよいためパルス幅が十分長く、タイミングの細かい調整なく、蹴り角のみで考えて良い。セプタム電磁石の蹴り角は、電流値と磁場の線形性を適当な係数であわせ制御している。蹴り角と磁場は以下の関係を持つ。
\begin{equation}
\theta(t_0) [mrad] = \frac{0.3B [T]l [m]}{E [GeV]}
\end{equation}
ここで、Bは磁場、lは有効磁場長、Eはエネルギーである。この関係から、磁場と電流値の関係は蹴り角と電流値の関係に直され、以前に測定された結果は、
\begin{eqnarray}
\theta_{S1} [mrad] &=& -0.018669 + 0.00012627 \times I [A] \nonumber \\
\theta_{S2} [mrad] &=& -0.013872 + 0.00012583 \times I [A] \nonumber
\end{eqnarray}
となっている。これを、セプタム電磁石のビーム応答に基づいた値に更新するためセプタム電磁石の電流値と蹴り角の関係を測定する。ビームから確認できるのは傾きの変化量であり、運転に用いている場所で磁場が飽和していないか確認した。有効磁場の長さ(1 \si{m}程度)によるビーム長の変化は無視すると、以下の関係に変換される。
\begin{eqnarray}
\Delta \theta_{S1} [mrad] = \Delta x'_{beam} [mrad] \nonumber \\
\Delta \theta_{S2} [mrad] = \Delta x'_{beam} [mrad] \nonumber
\end{eqnarray}
セプタム電流値のピーク値はセプタム電源のモニタ波形をオシロスコープのピークサーチで観測して求める。蹴り角は、セプタム電磁石の蹴り角を変化させて入射ビームの傾きの変化量を測定する。これらの結果から電流値と磁場の係数を校正する。セプタム電磁石のパラメータを表\ref{coef_Septum}にまとめる。

    \begin{table}[H]
    \caption{セプタム電磁石の設計値}
    \label{coef_Septum}
    \centering
    \begin{tabular}{ c c c c c } \hline
    Position & \multicolumn{2}{c}{Peak value} & \multicolumn{2}{c}{Setting value} \\ \cline{2-5}
     & [mrad] & [A] & [mrad] & [A] \\ \hline
    S1 & 166 & 6000 & 118.61 & 5344 \\
    S2 & 98  & 6500 & 93.69 & 6281 \\ \hline
    \end{tabular}
    \end{table}

\section{信号処理}
ここでは電極から返される生の電圧の時間波形をビーム位置に変化する手順とパルス波形に再構築する手順の説明する。初めにBPMの測定原理をまとめる。式の導出はOHOセミナーの資料\cite{Kuboki}を参照した。次に生の電圧の時間波形をビーム位置に変換するための解析手順をまとめる。最後にパルス波形へ再構築するための解析手順をまとめる。
%*******************************************************************************
\subsection{測定原理}
ビーム位置モニター(Beam Position Monitor:BPM)はビームの横方向の位置を測定する、モニターとして使用される。ここでBPMの電極はボタン電極、形状は円筒形を考える。線密度$\lambda$の電荷がボタンに誘起する電荷qは、
\begin{equation}
q = \frac{\lambda }{2\pi R} \int_{-\Delta \phi_0}^{-\Delta \phi_0} \frac{R^2-x^2}{R^2+x^2-2Rx(\varphi-\theta)} d\varphi
\end{equation}
で表される。ここで、x、$W << R$とすると、
\begin{equation}
\label{line_charge}
q = \frac{\lambda}{2\pi R} \left(1+\frac{2x}{R} \right)
\end{equation}
に計算される。軌道変位xが大きいときは式(\ref{line_charge})が示すように、誘導電荷と変位の関係は非線形になる。このことはビーム位置モニターとしてxの大きい領域まで使用する場合、x-q関係を補正する必要があることを示唆している。ボタン電極からの信号は、電荷量と電極までの距離に比例した電圧波形として同軸ケーブルに接続された処理回路まで伝搬される。電極部は伝搬系とインピーダンスマッチしておらず、ボタン周辺と真空チェンバーとの面から生じるキャパシタンス成分が強く、電荷の時間変動$dq/dt$の電流源で信号を発生する。したがってこれに接続されている50$\Omega$の同軸ケーブル入り口では等価回路の回路方程式より、
\begin{equation}
\label{eq_cir_capa}
\frac{dq}{dt}=C\frac{dV}{dt}+\frac{V}{R'}
\end{equation}
になる。発生する電圧をラプラス変換にて求めると、
\begin{equation}
\label{hpf_filter}
V(s) = \frac{q(s)}{C} \frac{s}{s+\frac{1}{CR} }
\end{equation}
となる。式(\ref{hpf_filter})からBPMは微分時定数$\tau = CR$のハイパスフィルター(HPF)型の周波数特性を示すことが予想される。バンチの幅が時定数より十分小さければ出力電圧はバンチの形を再現する。このモニターでは動作原理上、電極の長さLがバンチ幅より短くなければならない。

また、ボタン電極型の特徴として静電容量と時定数が小さい。したがって進行方向のバンチ幅が十分に短くないときビーム進行方向の電荷分布の影響を受ける。その分布を$f(t)$とすると、
\begin{equation}
\lambda(\tau) = Nef(\tau) = \frac{i_0}{f_0}f(\tau)
\end{equation}
で表される。ここで$f(\tau)$の積分値は1に規格化し、$f_0$、$\tau$はそれぞれ周回周波数、バンチ時間幅である。電極に誘起される全電荷は、
\begin{equation}
Q(t) = \frac{i_0W}{2\pi af_0\tau} \left(a+\frac{x}{2a} \right)[ <f> + \eta {f(\tau)-<f>}]
\end{equation}
である。ここで、$<f>$は$f(t)$の平均を示し、$\eta$はTransit time factorを示す。$\eta$は$\eta<1$かつ、電極の長さLとバンチ幅$\tau$の関数である。

ビームの分布関数を次のように仮定する。
\begin{equation}
f(t) = \frac{1}{t}\cos ^2(\frac{\pi t}{2\tau })
\end{equation}
すると、
\begin{eqnarray}
Q(t) &=& \frac{ i_0 WL}{ 4\pi acf_0\tau } \left( 1+\eta \cos \left(\frac{\pi t}{\tau} \right) \right) \\
\eta &=& \frac{ \sin (\theta_\tau /2)}{ (\theta_\tau /2)} \\
\theta_\tau &=& \frac{\pi L}{c \tau}
\end{eqnarray}
となる。回路方程式(\ref{eq_cir_capa})は解けて、出力波形は次のように求まる。
\begin{equation}
V(t) = -\frac{i_0WL\eta }{4\pi acf_0t}
\end{equation}
このモニターは電極の形が小さいので、真空チェンバー内に4個取り付けて、水平、垂直両方向の軌道変位を次の計算式により求めることができる。
\begin{equation}
\frac{ (V_A+V_D)-(V_B+V_C) }{ V_A+V_B+V_C+V_D } = U^i
\end{equation}
\begin{equation}
\frac{ (V_A+V_D)-(V_B+V_C) }{ V_A+V_B+V_C+V_D } = V^j
\end{equation}
位置感度係数kは円形チェンバーの場合x, y方向ともに同じ値になる。円筒型でないBPMの感度係数は有限要素法や境界要素法などの数値計算、もしくはワイヤ校正によって求める。通常、ワイヤー較正は、ビームサイズの小さい電子や陽子を観測する際に含まれる非線形効果を補正して精度よく位置を測定するために用いられる。

本研究では、xの大きい領域まで使用するためにワイヤー較正を使用した。BPMダクトの断面図を図\ref{figure_CSBPM}に示す。ダクトには4極の電極があり、伝搬ケーブルを接続されて各電極の信号比からビーム位置を導出できる。ビーム位置は、
\begin{eqnarray}
\label{BPM_calc}
x = \sum_{i=0}^3 \sum_{j=0}^3 k_x(i,j)U^iV^j \nonumber \\
y = \sum_{i=0}^3 \sum_{j=0}^3 k_y(i,j)U^iV^j
\end{eqnarray}
と表記される。kはモニターのマッピング情報から得られる非線形感度係数を表し、補正には表\ref{coef_BPM}のリストを使用した。\ref{coef_BPM}の非線形感度係数を使用した時のビーム位置マッピングは図\ref{calc_BPM}に示す。青点がビームダクト内をワイヤーを通して電流を流した位置、赤点は式(\ref{BPM_calc})で計算して測定される位置を示す。図\ref{calc_BPM}より、電極の外側を通過するビームは非線形応答によって縮退されることを確認できる。

     \begin{figure}[H]
        \centering
        \includegraphics[width=100mm]{./figure/BPM/CSbpm.png}
        \caption{BPMダクトの断面図.}
        \label{figure_CSBPM}
    \end{figure}

     \begin{figure}[H]
        \centering
        \includegraphics[width=100mm]{./figure/BPM/Plot_BPM.png}
        \caption{三次多項式フィットによる計算されたビーム位置}
        \label{calc_BPM}
    \end{figure}

    \begin{table}[H]
    \caption{BPMの非線形応答係数}
    \label{coef_BPM}
    \centering
    \begin{tabular}{ c  c c } \hline
    Coefficients & Horizontal & Vertical \\ \hline
    $k(0,0)$ & 0.000 & 0.000 \\
    $k(1,0)$ & 17.120 & 0.000 \\
    $k(1,1)$ & 0.000 & 16.432 \\
    $k(2,0)$ & 0.000 & 0.000 \\
    $k(2,1)$ & 0.000 & 0.000 \\
    $k(2,2)$ & 0.000 & 0.000 \\
    $k(3,0)$ & 18.829 & 0.000 \\
    $k(3,1)$ & 0.000 & -11.477 \\
    $k(3,2)$ & -27.138 & 0.000 \\
    $k(3,3)$ & 0.000 & 7.431 \\ \hline
    \end{tabular}
    \end{table}

%*******************************************************************************
\subsection{データ処理}
ボタン電極からの入力信号は、バンドパスフィルターを通過してADC変換された波形で観測される。図\ref{signal_start}に観測した入射ビームと蓄積ビームの信号を示す。縦軸のADC Countはビームの電荷量に相当する。横軸のSampleは入射トリガーが来てバッファー内に格納される信号を112 \si{MHz}でサンプリングした時間に相当する。バッファーに格納される範囲はメモリの容量で決まり、4096 Sampleまで記録される。

    \begin{figure}[H]
        \centering
        \includegraphics[width=100mm]{./figure/BPM/raw_signal.png}
        \caption{Libera Brilliance plusからのADC生データ.}
        \label{signal_start}
    \end{figure}

図\ref{signal_start}より、信号の周期は74 Samplesであることがわかる。これは自己相関関数を用いて相関の高い点の間隔を調べて求めた。
データ処理は、図\ref{signal_start}の信号から、オフセットを取り除き、1周期ごとに取り出して、ADC Countの絶対値を総和して、信号の大きさにする。この値を式(\ref{BPM_calc})に代入して位置を求めた。ビーム位置の取得は複数回行い、平均値を取得するが、入射ビームでは電荷量にばらつきがあるので次の例外処理を加えた。一例として図\ref{BPM_ex1}に入射振動の測定を100回行った結果を示す。左図の縦軸はビーム位置、右図の縦軸は電極からの信号の総和、横軸は測定回数に相当する。

    \begin{figure}[H]
        \centering
        \begin{tabular}{c}
        \begin{minipage}{0.45\hsize}
        \centering
        \includegraphics[clip, width=0.9\columnwidth]{./figure/BPM/bpm_ex2.png}
        \hspace{18mm} [1]水平方向のビーム位置
        \end{minipage}
        \begin{minipage}{0.45\hsize}
        \centering
        \includegraphics[clip, width=0.9\columnwidth]{./figure/BPM/bpm_ex3.png}
        \hspace{16mm} [2]電極からの信号強度
        \end{minipage}
        \end{tabular}
        \caption{入射振動の測定結果}
        \label{BPM_ex1}
    \end{figure}

図\ref{BPM_ex1}より、ビームのばらつきが信号の大きさに表れること、測定に失敗したデータも観測されること確認できる。図\ref{BPM_ex1}の右図にある、信号強度が少ないデータを除去した。

\subsection{データ収集タイミング}
データ収集タイミングはLinacでつくられた入射トリガを用いている。基本周波数はKEKBのマスターオシレータ510  \si{MHz}を分周した10.38546 \si{MHz}を位相同期回路の入力に用いて生成した571.2 \si{MHz}と、これを分周逓倍した10.39 \si{MHz}である\cite{LINAC_timing}。入射トリガーを受け取り、入力信号はバッファーに格納され、入射やキッカー電磁石の励磁が始まる。

はじめに記録が始まるタイミングと入射ビームのタイミングを調べる。測定は0.25 \si{nC}のビームを入射してボタン電極から返される信号を調べる。図\ref{signal_start}にボタン電極から入力される信号を示す。

    \begin{figure}[H]
        \centering
        \includegraphics[width=100mm]{./figure/BPM/start.png}
        \caption{入射振動の生データ.}
        \label{signal_start2}
    \end{figure}

図\ref{signal_start2}は図\ref{signal_start}の横軸を2000 Sampleまで拡大した図である。図\ref{signal_start2}からビームの入射タイミングがわかる。入射トリガとキッカー電磁石の励磁タイミングは正確で、タイミングを固定して(Sample=405)信号の抽出を行った。

次にトリガーとキッカーの励磁タイミングの同期をとるため計算と測定を行った。計算ではキッカー電磁石が半正弦波としてディレイタイムを変えた時のBPMで観測される振動を求める。測定ではディレイタイムを変えた時の蓄積ビームの振動を調べる。以下にK3の蹴り角を加えた時、ディレイタイムを周期時間まで掃引してビームの振動を計算した結果を示す。BPMの位置は以下の場所を使用した。

    \begin{figure}[H]
        \centering
        \includegraphics[width=110mm]{./figure/BPM/lattice.png}
        \caption{BPMの位置}
        \label{timing_lattice}
    \end{figure}

    \begin{figure}[H]
        \centering
        \includegraphics[width=110mm]{./figure/BPM/kick.png}
        \caption{K3に蹴り角を加えた時BPMで観測される振動(計算)}
        \label{oscil_K3}
    \end{figure}

図\ref{oscil_K3}の横軸の目盛りは周期時間である。縦軸はキッカーの蹴り角と各BPMで観測される振幅の比率を表す。図\ref{oscil_K3}の一周期目は単なる転送の応答であり、
\begin{equation}
\label{trans_single}
\begin{pmatrix}
x \\
x'
\end{pmatrix}_{BPM}
=
M(x_{BPM}|x_{K3})
\begin{pmatrix}
0 \\
\theta(t)
\end{pmatrix}_{Kicker}
\end{equation}
で表される。2周期(624から1248 nsec)の振動は前ターンの振動を含み、
\begin{equation}
\begin{pmatrix}
x_{n+1} \\
x'_{n+1}
\end{pmatrix}_{BPM}
=
M(x_{BPM}|x_{K3}) \left \{
\begin{pmatrix}
x \\
x'
\end{pmatrix}_{K3}
+
\begin{pmatrix}
0 \\
\theta(t)
\end{pmatrix}_{K3}
\right \}
\end{equation}
で記述される。3周期以降は、蹴り角が含まれないので応答は一周の転送行列を重ねて、
\begin{equation}
\begin{pmatrix}
x_{n+1} \\
x'_{n+1}
\end{pmatrix}_{BPM}
=
M(x_{BPM}|x_{K3})
\begin{pmatrix}
x_n \\
x'_n
\end{pmatrix}_{K3}
\end{equation}
となる。次にディレイタイムを掃引した時の蓄積ビームの振動を測定して平均値をプロットしたものを示す。

    \begin{figure}[H]
        \centering
        \includegraphics[width=140mm]{./figure/BPM/bpm.png}
        \caption{K3に蹴り角を加えた時BPMで観測される振動(実測)}
        \label{timing_kick}
    \end{figure}

図\ref{timing_kick}の囲った横軸の範囲が周回時間に相当する。3つの線のプロットは1周期違いのプロットであり、計算で求めた0から624の範囲に$\times3$回重ね書きしたものに相当する。2つの結果を比べるため、図を整形して比較した。これを図\ref{com_kick}に示す。

    \begin{figure}[H]
        \centering
        \includegraphics[width=140mm]{./figure/BPM/K3_tim.png}
        \caption{K3に蹴り角を加えた時BPMで観測される振動の比較}
        \label{com_kick}
    \end{figure}

図\ref{com_kick}のエラーバー付きは図\ref{timing_kick}の結果であり、標準偏差を示す。実線は計算値を示す。波形の立ち上がりの部分では半正弦波で計算できているが、2周期目、3周期目と違いが大きくなっていることがわかる。これはパルス幅が周回時間より長く、反射の影響を含むことを表している。タイミングの合わせ方は、図\ref{timing_kick}の切り出し範囲を調整することで行った。同様の操作を全てのキッカーに対して行っている。
%*******************************************************************************
\section{測定結果\label{meas_result_inj}}
\subsection{入射点でのビーム位置関係}
入射ビームの位置の基準となる蓄積ビームの中心軌道の測定は3 \si{mA}のシングルバンチで行った。BPMからの振動は電極からの生信号を解析して、ビーム位置として求めている。
結果を表\ref{PS_stored_ID02}にまとめる。

    \begin{table}[H]
    \caption{直線部2番での蓄積ビームの位相空間情報}
    \label{PS_stored_ID02}
    \centering
    \begin{tabular}{ l c c } \hline
    & Position [mm] & Angle [mrad] \\ \hline
    Horizontal & $-1.83\pm0.28$ & $0.52\pm0.09$ \\ 
    Vertical    & $2.10\pm0.17$ & $-0.07\pm0.07$ \\ \hline
    \end{tabular}
    \end{table}

入射ビームの測定は、1 \si{nC}の入射バンチを1 \si{Hz}入射して、電極からの信号をターンごとに切り分けて、入射した瞬間を抽出して行った。図\ref{PS_inj_pt}に入射バンチを100点プロットした結果を示す。横軸はビーム位置、縦軸は角度の位相空間情報を表す。表\ref{PS_inj_ID02}は\ref{PS_stored_ID02}の蓄積ビームの座標からとった相対座標をまとめた。

     \begin{figure}[H]
        \centering
        \includegraphics[width=100mm]{./figure/04_Measurement/PS_inj_pt.png}
        \caption{入射ビームの位相空間プロット}
        \label{PS_inj_pt}
    \end{figure}

    \begin{table}[H]
    \caption{入射点での入射ビームの相対位相空間情報}
    \label{PS_inj_ID02}
    \centering
    \begin{tabular}{ l c c } \hline
    & Position [mm] & Angle [mrad] \\ \hline
    Horizontal & $30.5\pm3.20$ & $3.11\pm0.42$ \\ 
    Vertical    & $0.06\pm0.23$ & $102\pm0.23$ \\ \hline
    \end{tabular}
    \end{table}

図\ref{PS_inj_pt}のプロットに信号強度の削れた入射ビームは測定されていないと仮定する。このとき水平位置の分布の最小値はセプタム外壁と入射ビームのばらつきを足した距離に相当する。水平位置分布の最小値からバンプハイトとセプタム壁までの距離2 \si{mm}とセプタム壁の厚み5 \si{mm}を引いた距離は約$17.5+\Delta x_{inj} \si{mm}$である。これは測量で得たバンプハイトの19 \si{mm}と一致している、ビームベース測定で入射ビームの測定が可能であることを確認した。表\ref{PS_inj_ID02}の水平の位相空間座標は、設計モデルの (27 \si{mm}, 2 \si{mrad})から初期振幅の増大が確認できる。
%*******************************************************************************
\subsection{キッカー電磁石}
キッカー電磁石のパルス波形の測定は、4台のパルス電磁石を独立に1台ずつ0.4 \si{mrad}の蹴り角を設定し蓄積ビームの応答を測定した。ビームの応答はパルス波形に再構築するため、励磁のタイミングを遅延時間方向に25 \si{nsec}刻みで掃引して0 \si{nsec}から2500 \si{nsec}までのパルス波形をプロットした(図\ref{PS_KK_re})。0.4 \si{mrad}はキッカーが正しく動作する最小の蹴り角である。

     \begin{figure}[H]
        \centering
        \includegraphics[width=90mm]{./figure/04_Measurement/PulseShape_KK_recon.png}
        \caption{ビームの応答から求めたキッカーのパルス波形(0.4 mrad).}
        \label{PS_KK_re}
    \end{figure}

図\ref{PS_KK_re}よりパルス波形より蹴り角を揃え、タイミングを実際のビームの挙動から整列できた。また、パルス幅 (Zero-to-Zero)がキッカーによって異なるため原理的にキッカーバンプマッチングを取れないことを確認できる。K1の測定結果はID02からの位相進みが振幅の節となるため、ID01から転送した結果を採用した。精度をあげるためには、位相進みが振動の腹となる、直線部を選択すればよい。次に、キッカー電磁石K3をインストール前に行われた磁場測定と比較した(図\ref{Com_KK})。

     \begin{figure}[H]
        \centering
        \includegraphics[width=100mm]{./figure/04_Measurement/compare_KK.png}
        \caption{磁場測定とビームベース測定の比較}
        \label{Com_KK}
    \end{figure}

図\ref{Com_KK}の立ち上がりから前半部分までパルス波形は一致している。しかし、設定振幅の蹴り角が反対方向にはみ出す反射以降で違うことがわかる。この差異はあらゆる誤差を含んだ現実の加速器を対象とする、リング内の電磁石から制御系までの環境を含んだ効果である。この結果を用いたシミュレーションを次章で示す。

%*******************************************************************************
\subsection{セプタム電磁石}
セプタム電磁石に対しては制御で使われる電流値の線形性を測定した。これはオシロスコープで観測される電流波形のピーク値と制御系からリードバックされる値の関係性を示す。電流値は0 \si{A}から設計の最大値まで変化させてモニターの蹴り角を調べた。結果はオシロスコープのピークサーチの分解能によるばらつきを考慮すると、最小二乗法による回帰分析の結果から性能の最大値まで線形性をもつことを確認した(図\ref{pre_septum})。このときの蹴り角と電流の関係を2004年に測定した結果と比較して表\ref{table_pre_septum}にまとめる。

    \begin{figure}[H]
        \centering
        \begin{tabular}{c}
        \begin{minipage}{0.45\hsize}
        \centering
        \includegraphics[clip, width=0.9\columnwidth]{./figure/04_Measurement/linear_S1_pre.png}
        \hspace{16mm} [1]S1
        \end{minipage}
        \begin{minipage}{0.45\hsize}
        \centering
        \includegraphics[clip, width=0.9\columnwidth]{./figure/04_Measurement/linear_S2_pre.png}
        \hspace{16mm} [2]S2
        \end{minipage}
        \end{tabular}
        \caption{セプタム電磁石の励磁特性}
        \label{pre_septum}
    \end{figure}


    \begin{table}[H]
    \caption{セプタム電磁石の励磁特性}
    \label{table_pre_septum}
    \centering
    \begin{tabular}{ l c c c } \hline
    Position & Measurement & Setting value at 2004 & \% \\ \hline
    S1 & 0.022269 & 0.022827 & -2.444 \\ 
    S2 & 0.016564 & 0.015182 & +9.103 \\ \hline
    \end{tabular}
    \end{table}

次に図\ref{pre_septum}で測定された傾きの大きさがビームベースの結果と一致するか確認する。測定はセプタム電磁石上流にある補正電磁石で入射ビームが直線部のダクト中心を通るよう調整した後、モニターでセット値から$\pm$0.3 mrad変化させたときの傾きの大きさを調べた。結果を図\ref{septum}に示す。この測定は入射できないと測れないので0から最大値までの測定はできない。セプタム電磁石の有効磁場長内で変化する位置は小さいと無視すると、セプタムで蹴られたビームはドリフト空間の転送の効果だけを経て、蹴られた角度がそのまま観測される。オシロスコープのピークサーチから返される電流値の値は図\ref{pre_septum}の結果を用いてモニターの電流値からセット値を推定した。左図の横軸のセット値は2004年の磁場測定の結果を元に-2.5\%の縮小をとり、右図の横軸のセット値は+9.5\%の拡大をとる。縦軸はビームで測定した100点の位相空間から平均と標準偏差のエラーバーをプロットした。

    \begin{figure}[H]
        \centering
        \begin{tabular}{c}
        \begin{minipage}{0.45\hsize}
        \centering
        \includegraphics[clip, width=0.8\columnwidth]{./figure/04_Measurement/linear_S1.png}
        \hspace{16mm} [1] S1(114.8 mrad)
        \end{minipage}
        \begin{minipage}{0.45\hsize}
        \centering
        \includegraphics[clip, width=0.8\columnwidth]{./figure/04_Measurement/linear_S2.png}
        \hspace{16mm} [2] S2(93.7 mrad)
        \end{minipage}
        \end{tabular}
        \caption{セプタム電磁石の蹴り角の変化量}
        \label{septum}
    \end{figure}

セプタムの蹴り角を0.3 \si{mrad}変えると直線部を通るビームはダクトの端近くを通るため非線形効果が大きくなる、そのため測定条件を固定して変化量は大きくできなかった。補正電磁石の測定も実施したが同様の理由で測定精度が得られていない。以上の操作から得た、電流値-磁場の係数を表\ref{Parameters_Septum2}にまとめる。

    \begin{table}[H]
    \caption{セプタム電磁石の励磁係数}
    \label{Parameters_Septum2}
    \centering
    \begin{tabular}{ c c c c } \hline
    Name & Prev. & Update & \\
            & [mrad/A] & [mrad/A] & \% \\ \hline
    S1 & 22.2 & 22.4 & -1.6 \\
    S2 & 16.6  & 17.1 & 12.94 \\ \hline
    \end{tabular}
    \end{table}

表\ref{Parameters_Septum2}からS1とS2どちらも磁場が飽和することなくビームを蹴っていることがわかる。BTの電磁石を調整するためにはビームベース測定で電磁石の応答を調べることは困難で、入射ビームのベータトロン振動を直接観測して小さくする調整が必要となる。

%*******************************************************************************
\chapter{入射シミュレーション\label{simulation}}
本章では、前章で求めた入射パラメータを使用し、入射率の改善と蓄積ビームの振動の抑制という入射要件を満たしたパラメータの組み合わせを計算する。入射率は加速器シミュレーションコードSADで多粒子追跡を行い評価した\cite{SAD}。蓄積ビームの振動は、キッカー電磁石の蹴り角に偏差を導入して生じるベータトロン振動の大きさ(キッカーバンプマッチング)で評価した。作成したモデルの妥当性はビームスタディの結果と比較して調べる。
%*******************************************************************************
\section{キッカーバンプ}
ビームベース測定で得られた入射パラメータを使用して入射のシミュレーションを行う。はじめに理想的なシングルターン入射の計算を考える。通常の入射でキッカー電磁石は、上流2台(K1, K2)と下流2台(K3, K4)の4台のパルス電磁石を用いて局所バンプを形成している。ここでバンプの計算を簡単にするためK1、K3、K4の3台でバンプを形成する。K2は不調で使われていない。K1の蹴り角を決めると、バンプが閉じるK3、K4の蹴り角の組み合わせは一意に決まる。上流のキッカー電磁石に求められる要件は、入射ビームをアクセプタンス内に収めるためにセプタム壁の近くまで寄せることであった。バンプハイトを決めると上流の蹴り角が決まる。ここでバンプハイトを17 \si{mm}とするK1の蹴り角を求め、K3、K4の蹴り角は調整していないときの軌道の例を図\ref{bump_height}にプロットする。17 \si{mm}は蓄積ビームがセプタムのアブソーバにあたらない最大蹴り角である。

    \begin{figure}[htbp]
        \centering
        \includegraphics[width=100mm]{./figure/05_Simulation/bump_height.png}
        \caption{バンプハイトを17 mmとした時のリング1周の軌道}
        \label{bump_height}
    \end{figure}

図\ref{bump_height}ではK4の下流で蓄積ビームが振動している。下流のキッカーは上流のキッカーと組み合わせてバンプが閉じるように設定される。次に下流のキッカーの蹴り角をバンプが閉じるようにフィッティングする。フィッティングの計算ではK4の場所で位置と運動量のずれがなくされる。このとき得られたバンプの形状を図\ref{single_bump}に示す。

    \begin{figure}[H]
        \centering
        \includegraphics[width=100mm]{./figure/05_Simulation/single_bump.png}
        \caption{シングルターン入射時の閉じたバンプ軌道}
        \label{single_bump}
    \end{figure}

この図\ref{single_bump}のバンプの形状が通常のシングルターン入射で使用される。この計算は全てのキッカー電磁石でパルス波形が一致し、設定値の蹴り角に揃えられ、励磁タイミングが同期されていることが仮定されている。しかし、入射パラメータの測定結果から、キッカー電磁石のパルス幅は異なり、反射を含むため、複数回ビームを蹴っていることが予想された。これを考慮したバンプを形成するためのマルチターンキック入射の場合を考える。2ターン目のキッカー電磁石の蹴り角は測定したパルス波形を多項式フィットしてシミュレーションに導入した(図\ref{PS_KK})。また、単純のためパルスタイミングは入射の瞬間にピーク位置で同期され、1ターン目と2ターン目の蹴り角でマルチターンキックを考える。バンプ軌道をシングルターン入射と同じ蹴り角で行い、蓄積ビームの振動を2周目まで追いかけて計算を行った結果を図\ref{bump_second}に示す。

    \begin{figure}[H]
        \centering
        \includegraphics[width=100mm]{./figure/05_Simulation/PulseShape_KK.png}
        \caption{計算に導入したキッカー波形}
        \label{PS_KK}
    \end{figure}

    \begin{figure}[H]
        \centering
        \includegraphics[width=100mm]{./figure/05_Simulation/second_bump.png}
        \caption{リング2周分の蓄積ビームの軌道}
        \label{bump_second}
    \end{figure}

図\ref{bump_second}から2ターン目のキッカー電磁石の蹴り角によって、K4以降に振動が励起されているとわかる。これを打ち消すために1ターン目の蹴り角を調整して、2ターン目以降の偏差を消すフィッティングを行った。キッカーのパルス波形は、1ターン目のタイミングと蹴り角が決定されると2ターン目の蹴り角は決められる。その蹴り角の対応関係は変化しないので、固定されたタイミングではバンプを形成することができる。このときのバンプ軌道と計算結果を示す。

    \begin{figure}[H]
        \centering
        \includegraphics[width=100mm]{./figure/05_Simulation/second_bump2.png}
        \caption{マルチターン入射時の閉じたバンプ軌道}
        \label{bump_second2}
    \end{figure}

以上の計算によって、マルチターン入射時のキッカー電磁石の蹴り角を決定した。ここでは、バンプの高さは17 mmとして求めた蹴り角の組み合わせの表\ref{parameter_KK}を記す。0ターン目、3ターン目を含んだマルチターンキックの計算も同様にして行われる。

    \begin{table}[htbp]
    \caption{マルチターンとシングルターンでの蹴り角の組み合わせ(x=17)}
    \label{parameter_KK}
    \centering
    \begin{tabular}{ c  c c } \hline
     & multi turn injection (1-2 turns) & single turn injection \\ \hline
    1 turn & & \\
    $K1.1$ & -2.630 & -2.630 \\
    $K2.1$ & 0.000 & 0.000 \\
    $K3.1$ & -1.665 & -1.939 \\
    $K4.1$ & -3.224 & -3.675 \\
    2 turn & & \\
    $K1.2$   & -1.657 &  \\
    $K2.2$   & 0.000 &  \\
    $K3.2$   & -0.498  &  \\
    $K4.2$   & -1.183 &  \\ \hline
    \end{tabular}
    \end{table}

%*******************************************************************************
\newpage
\section{キッカーバンプマッチング}
実験結果からキッカー電磁石は、励磁タイミングやパルス幅、反射などの誤差を含み、バンプが閉じない原因となることを確認した。またバンプ内に6極電磁石があるため、ピーク位置以外のタイミングでは誤差が生じる。この時の振動は、ピーク位置から前後のタイミングで単粒子追跡して調べるとわかる。この振動をリング一周のBPMで観測し、自乗平方和した値を誤差の評価値とする。これは以下の式(\ref{objective_function})で表される。
\begin{equation}
\label{objective_function}
R_{single} = \sqrt{ \frac{1}{N} \sum_{j}^{turn} \sum_{i}^{BPM}x_i^2 }
\end{equation}
式(\ref{objective_function})のiおよびjはBPM番号と測定ターンを表す。マルチバンチは式(\ref{objective_function})にディレイタイムを加えて、
\begin{equation}
\label{objective_function2}
R_{multi} = \sum_{t}^{delay} R_{single}
\end{equation}
と表される。タイミングを変えて単粒子追跡した図\ref{eff_cod}に示す。また、多項式フィッティングから求めたバンプ波形を用いてタイミングがピーク位置から変化したときの蹴り角の誤差量を求めた。これを図\ref{err_KK}に示す。

    \begin{figure}[H]
        \centering
        \includegraphics[width=120mm]{./figure/05_Simulation/cod2.png}
        \caption{K3のディレイを変えた時のリング4周分の蓄積ビームの軌道}
        \label{eff_cod}
    \end{figure}

    \begin{figure}[H]
        \centering
        \includegraphics[width=90mm]{./figure/05_Simulation/err_kk_bump_calc.png}
        \caption{キックエラーの計算値}
        \label{err_KK}
    \end{figure}

図\ref{eff_cod}と図\ref{err_KK}からピーク位置のタイミングと蹴り角では振動されないが、ピーク位置から離れるほど振幅を大きくして振動することがわかる。この振動をピーク位置から前後半周期の範囲を掃引して足した値が式(\ref{objective_function})である。つまり、キッカーバンプの残留振動を自乗平方和した値、蓄積ビームの振動量をここで評価値と呼ぶ。この式で求まる値を最小化してキッカーバンプの形成とピーク位置以外のタイミングに対するマッチングの調整を行う。計算は簡単のためバンチはリングの中で均一に整列されると仮定した。目標値の変化をK3とK4のディレイを連続的に変えて目標値の大きさをプロットしたものを図\ref{Err_obj1}に示す。

    \begin{figure}[H]
        \centering
        \includegraphics[width=70mm]{./figure/05_Simulation/Err_obj1.png}
        \caption{K3,K4のディレイを変えた時の評価値の振舞い(1から2ターンのマルチターン入射)}
        \label{Err_obj1}
    \end{figure}

図\ref{Err_obj1}の縦軸、横軸は下流のキッカーのディレイでありパルス波形のピーク位置を0 \si{nsec}とした。カラーバーの範囲は評価値を表す。これを周期構造へ拡張して評価する。これを図\ref{Err_obj2}で示す。

    \begin{figure}[H]
        \centering
        \includegraphics[width=150mm]{./figure/05_Simulation/Err_obj2.png}
        \caption{周期構造への拡張}
        \label{Err_obj2}
    \end{figure}

ここまでは1から2ターンまでの入射条件を考えた。次に現実の加速器モデルを反映したマルチターン入射への拡張を考える。はじめにビームを使った実験で目標値はどのように振る舞うか調べた。この結果を計算モデルと比較してマルチキックの条件を決定する。1から2ターンで計算したマッピングと同様に計算して0から2ターンと0から3ターンまでの場合を図\ref{Opt_range}に示す。パルス波形は多項式でフィットしているのでデータの含まれない0 \si{nsec}以前と2500 \si{nsec}以降の蹴り角は除外した。キッカーパラメーターは表\ref{Opt_prev}の値を使用した。

    \begin{table}[htbp]
    \caption{測定時のキッカーパラメータ}
    \label{Opt_prev}
    \centering
    \begin{tabular}{ c  c c } \hline
     & Angle [mrad] & Delay time [nsec] \\ \hline
    $K1.1$   & 2.22 & 1900 \\
    $K2.1$   & 0.00 & 1800 \\
    $K3.1$   & 1.51 & 1720 \\
    $K4.1$   & 3.57 & 2400 \\ \hline
    \end{tabular}
    \end{table}

一般的にマルチターンキックの影響は入射条件の制限として現れる。これは図\ref{Opt_single}の比較より反射を含めた効果より立ち上がり、ピーク位置、立下りの3回で蹴り角を加えている効果で強く制限されていることがわかった。目標値の振る舞いを周期構造に変換したものを図\ref{Opt_range}に示す。図\ref{Opt_range}ではK3の励磁タイミングを早くしてK4の励磁タイミングを遅くすることで小さくできることでマッチングされると計算された。このマッピングがビームスタディと一致するか比較した。

    \begin{figure}[H]
        \centering
        \begin{tabular}{c}
        \begin{minipage}{0.45\hsize}
        \centering
        \includegraphics[clip, width=1.0\columnwidth]{./figure/06_Optimization/TK02.png}
        \hspace{16mm} [1] 0 - 2 turn
        \end{minipage}
        \begin{minipage}{0.45\hsize}
        \centering
        \includegraphics[clip, width=1.0\columnwidth]{./figure/06_Optimization/TK03.png}
        \hspace{16mm} [2] 0 - 3 turn
        \end{minipage}
        \end{tabular}
        \caption{目標値の振る舞い(シングルバンチ)}
        \label{Opt_single}
    \end{figure}

    \begin{figure}[H]
        \centering
        \begin{tabular}{c}
        \begin{minipage}{0.45\hsize}
        \centering
        \includegraphics[clip, width=1.0\columnwidth]{./figure/06_Optimization/range02.png}
        \hspace{16mm} [1] 0 - 2 turn
        \end{minipage}
        \begin{minipage}{0.45\hsize}
        \centering
        \includegraphics[clip, width=1.0\columnwidth]{./figure/06_Optimization/range03.png}
        \hspace{16mm} [2] 0 - 3 turn
        \end{minipage}
        \end{tabular}
        \caption{周期構造を含んだ目標値の振る舞い(マルチバンチ)}
        \label{Opt_range}
    \end{figure}

ビームスタディは10 \si{mA}のマルチバンチのビームを使い、蓄積ビームの振動を測定した。振動は2台のBPMで測定する。ターン数はキッカーが結果を図\ref{Opt_mul}の左側に示す。また5 \si{mA}のシングルバンチのビームを使って同様の測定方法で調べた結果を右図に示す。測定の開始タイミングは、入射トリガーがきてキッカー電磁石が励磁された後、最初の数十ターンを除いたタイミングを使う。目標値は200ターン分のターンごとの振動をRMSにして求めた。カラーバーの範囲は最大-最小値の範囲で正規化している。

    \begin{figure}[H]
        \centering
        \begin{tabular}{c}
        \begin{minipage}{0.45\hsize}
        \centering
        \includegraphics[clip, width=1.0\columnwidth]{./figure/06_Optimization/sin.png}
        \hspace{16mm} [1] single bunch
        \end{minipage}
        \begin{minipage}{0.45\hsize}
        \centering
        \includegraphics[clip, width=1.0\columnwidth]{./figure/06_Optimization/mul.png}
        \hspace{16mm} [2] multi bunch
        \end{minipage}
        \end{tabular}
        \caption{周期構造を含んだ目標値の振る舞い(測定結果)}
        \label{Opt_mul}
    \end{figure}

    \begin{figure}[H]
        \centering
        \begin{tabular}{c}
        \begin{minipage}{0.45\hsize}
        \centering
        \includegraphics[clip, width=1.0\columnwidth]{./figure/06_Optimization/TK03_ex.png}
        \hspace{16mm} [1] single bunch
        \end{minipage}
        \begin{minipage}{0.45\hsize}
        \centering
        \includegraphics[clip, width=1.0\columnwidth]{./figure/06_Optimization/range03_ex.png}
        \hspace{16mm} [2] multi bunch
        \end{minipage}
        \end{tabular}
        \caption{周期構造を含んだ目標値の振る舞い(計算結果)}
        \label{Opt_ex}
    \end{figure}

以上の結果を比べて、入射パラメータの測定で得られた結果は近いことが期待できる。またマルチターンキックの影響は0から3ターンまで粒子追跡をすると反映できると分かった。実ビームの結果である図\ref{Opt_mul}から、目標値は連続的で最適化が可能であることが確認できる。

%*******************************************************************************
\section{入射振動}
バンプを形成する際、タイミングの固定されたマルチターン入射時で蓄積ビームを振動させないキッカー電磁石の蹴り角の組み合わせは一意に決まることを前節で示した。その時の入射率は、ビームの位置関係に依存する。入射振幅を抑えるため入射ビームの位置はバンプ軌道にできるだけ寄せられるがセプタム壁の厚みと入射ビームの分布の広がりから制限されてビーム位置$x_{pos}$は図\ref{cross_inj}のように決められる。

     \begin{figure}[H]
        \centering
        \includegraphics[width=100mm]{./figure/05_Simulation/cross_inj.png}
        \caption{入射点での位相空間のプロット}
        \label{cross_inj}
    \end{figure}

ここで、$W_{acc}$は設計軌道からセプタム壁までの距離、$W_{blade}$はセプタム壁の厚み、$W_{inj}$はセプタム壁から入射ビームまでの距離、$\Delta x$は入射ビームのばらつきに相当する。水平方向の角度についてはセプタム電磁石で調整される。図\ref{bump_inj2}にマルチターンキック入射でバンプを小さくなるようにフィティングした結果を示す。セプタム電磁石の調整によって入射振動は若干抑えられていることがわかる。

    \begin{figure}[htbp]
        \centering
        \includegraphics[width=100mm]{./figure/05_Simulation/second_inj.png}
        \caption{セプタム補正前の入射ビーム軌道}
        \label{bump_inj}
    \end{figure}

    \begin{figure}[H]
        \centering
        \includegraphics[width=100mm]{./figure/05_Simulation/second_inj2.png}
        \caption{セプタム補正後の入射ビーム軌道}
        \label{bump_inj2}
    \end{figure}

以上の操作によって、局所バンプの高さを決めた時のキッカー電磁石の蹴り角の組み合わせと入射ビームの水平方向の位相空間情報が決定される。垂直方向の位相空間情報は原点に、縦方向の位相空間情報はシンクロトロン入射のためエネルギー偏差を加えて、これを入射の始条件とする。
%*******************************************************************************
\section{オプティックスの測定}
本節ではベータトロン関数の測定について記述する。ベータトロン関数は横方向の振動の大きさを表し、入射率の計算に必要なパラメータである。測定方法ではLinear Optics Debugging Code LOCO (Linear Optics from Closed Orbits)の手法を説明する\cite{LOCO}。測定結果ではLOCOの解析過程と結果をまとめる。ベータトロン関数の測定原理はOHOセミナーの資料\cite{Kobayashi_oho}を主に参照した。
\subsection{測定原理}
ベータトロン関数の直接的な測定は、四重極強度を微小変化させてチューンシフトを測定する方法とベータトロン振動の振幅から測定する方法がある。間接的な測定は、複数のチューンの位相差を調べて比をとり求める方法\cite{castro}や応答行列を軌道フィッティングして求める方法\cite{LOCO}などがある。

はじめにベータトロン関数の直接的な測定についてまとめる。チューンシフトから求める方法は以下の式、
\begin{equation}
\Delta\nu = \frac{1}{4\pi} \int_0^s K(s)\beta(s) ds
\end{equation}
で与えられる。これを整理してベータトロン関数は、
\begin{equation}
\beta = \frac{4\pi\Delta\nu}{K(s)L}
\end{equation}
で求められる。ここでLは4極電磁石の磁石長である。この方法では式にK値を含み、磁石の履歴をうけるため正確な測定は難しい。ベータトロン振動の振幅から測定する方法はステアリング強度$\Delta \theta$を変化させて生じるCOD (closed orbit distortion)、
\begin{equation}
\label{LOCO}
x_{COD}(s) = \frac{\sqrt{\beta(s_1)\beta(s_0)}}{2\sin (\pi \nu)}\cos (\pi \nu-|\phi(s_0)-\phi(s_1)|)\Delta \theta
\end{equation}
で与えられる。これを整理して位相関係が小さいときベータトロン関数は、
\begin{equation}
\beta(s_1) = \frac{2x_{COD}tan(\pi \nu)}{\Delta \theta}
\end{equation}
で求められる。この方法では、ターンごとのビーム位置モニターを用いて、既知のチューンとステアリング強度の変化量$\Delta \theta$があって、振幅からベータトロン関数$\beta(s)$が求められる。測定精度はBPMとステアリング強度の精度とBPMとステアリングまでの位相関係に影響される。

つぎにベータトロン関数の間接的な測定についてまとめる。間接的な測定の利点は、ベータトロン関数を複数のパラメータから測定できる点にある、パラメータを変えながら複数回測定して平均化することによって測定の堅牢性を確保できる。測定は位相差から求める方法や軌道フィッティングから求める方法などがあるが、PFでは任意の場所に設置して位相差を調べることができる高速BPMは小数であるため、軌道フィッティングから求めるLOCOを採用した。LOCOについて説明する。LOCOとは蓄積リングの補正電磁石を微小な角度で蹴り、BPMで観測される式(\ref{LOCO})の振幅の応答行列からフィッティングを行って、加速器パラメータを推定する手法である。推定は軌道レスポンスと実測のレスポンスの残差が最小となるように軌道をフィッティングする。残差の式は、
\begin{equation}
\label{obj_LOCO}
f \left(\Delta K \right) = \chi ^2 = \sum_{i,j}\frac{ \left(R^{meas}_{ij}-R^{model}_{ij} \left(\Delta K \right) \right)^2}{\sigma_i^2}
\end{equation}
である。式(\ref{obj_LOCO})中のiおよびjはそれぞれBPM番号とステアリング番号を表す。$\sigma$はBPMのノイズレベルでありフィッティング時の重みづけの役割を果たす。測定された応答行列は以下の変換が加えられる。BPMゲインは実際のビームの位置$x_{beam}$に対して、
\begin{equation}
\begin{pmatrix}
x_{meas} \\
y_{meas}
\end{pmatrix}
=
\begin{pmatrix}
g_x & c_x \\
c_y & g_y
\end{pmatrix}
\begin{pmatrix}
x_{beam} \\
y_{beam}
\end{pmatrix}
\end{equation}
の変換を加える。ここで、gはBPMゲイン、cはカップリングの大きさを表す。また、$|c_x-c_y|$のクランチは理想的な形状からのずれを表す。ステアリング強度も同様にゲインとクランチが振動に対して、
\begin{equation}
\begin{pmatrix}
\theta_{x,meas} \\
\theta_{y,meas}
\end{pmatrix}
=
\begin{pmatrix}
g_{kx} \\
c_{kx}
\end{pmatrix}
\theta_{x,meas}
\end{equation}
の変換を加える。このとき、観測される応答行列は、
\begin{eqnarray}
R_{i,j}^{meas} &=& \frac{\Delta x^{meas}}{\theta_x^{meas}} \nonumber \\
&=& \frac{g_x\Delta x^{beam}}{\theta_x^{beam}/g_{kx}} \nonumber \\
&=& g_xg_{kx}R_{xx,ij}^{beam}
\end{eqnarray}
で示される。BPMとステアリング強度のゲインは分離して測定でき固定することで式(\ref{obj_LOCO})は非線形の最小化問題を解くのと同様になる。この最小化問題を解くことでベータトロン関数を求められる。

%*******************************************************************************

\subsection{測定結果}
応答行列の測定は非線形効果を最小限に抑えるためステアリングの強度を水平で0.2 \si{A}、垂直で0.1 \si{A}とした。これによってCODを300 \si{\mu m}におさめている。また再現性を確保するため5回の初期化を挟む。BPM感度係数とステアリング強度のゲインは解を固定するため定数とした。LOCOで用いたパラメータを表\ref{loco_parameter}にまとめる。初めに2極成分をフィッティングした。結果を図\ref{loco_horizontal_dipole}と図\ref{loco_vertical_dipole}に実測のレスポンス$R_{meas}$とモデルのレスポンスの差分$R_{meas}-R_{model}$で示す。ここでクランチの影響は小さいとして無視している。

    \begin{table}[H]
    \caption{LOCOパラメータ}
    \label{loco_parameter}
    \begin{center}
    \begin{tabular}{ c c c c } \hline
    名称 & & 単位 & 個数 \\ \hline
    水平ステアリング & $\Delta \theta$ & mrad/A & 28 \\
    垂直ステアリング & $\Delta \theta$ & mrad/A & 42 \\ 
    4極磁場強度 & & $m^{-2}$ & 72 \\
    4極スキュー & & mrad  & 14 \\ \hline
    \end{tabular}
    \end{center}
    \end{table}

    \begin{figure}[H]
        \centering
        \begin{tabular}{c}
        \begin{minipage}{0.45\hsize}
        \centering
        \includegraphics[clip, width=1.0\columnwidth]{./figure/LOCO/meas_h.png}
        \hspace{16mm} [1]実測レスポオンス
        \end{minipage}
        \begin{minipage}{0.45\hsize}
        \centering
        \includegraphics[clip, width=1.0\columnwidth]{./figure/LOCO/model_h.png}
        \hspace{16mm} [2]実測レスポンス-測定レスポンス
        \end{minipage}
        \end{tabular}
        \caption{水平方向の応答行列}
        \label{loco_horizontal_dipole}
    \end{figure}

    \begin{figure}[H]
        \centering
        \begin{tabular}{c}
        \begin{minipage}{0.45\hsize}
        \centering
        \includegraphics[clip, width=1.0\columnwidth]{./figure/LOCO/meas_v.png}
        \hspace{16mm} [1]実測レスポオンス
        \end{minipage}
        \begin{minipage}{0.45\hsize}
        \centering
        \includegraphics[clip, width=1.0\columnwidth]{./figure/LOCO/model_v.png}
        \hspace{16mm} [2]実測レスポンス-測定レスポンス
        \end{minipage}
        \end{tabular}
        \caption{垂直方向の応答行列}
        \label{loco_vertical_dipole}
    \end{figure}

次に4極成分をパラメータに加えてフィッティングした。4極成分のフィッティングはベータ関数、分散関数、チューン、位相進み変えるパラメータで更新は以下の式で行う。

\begin{equation}
u = \left( -\frac{\Delta x^2}{\Delta p_1},-\frac{\Delta x^2}{\Delta p_2},-\frac{\Delta x^2}{\Delta p_3},\cdots,-\frac{\Delta x^2}{\Delta p_n} \right)
\end{equation}

図\ref{loco_H}にフィッティングの結果を示す。縦軸は残差の大きさ、横軸は最急降下法の反復回数を表す。この調整では残差の変化量に相当する偏微分の値が最も大きい方向にパラメータを更新している。

    \begin{figure}[H]
        \centering
        \includegraphics[width=90mm]{./figure/LOCO/fit_loco_q.png}
        \caption{4極磁場成分のフィッティング.}
        \label{loco_H}
    \end{figure}

最後にスキュー成分をパラメータに加えてフィッティングした。図\ref{loco_skew}はレスポンスの残差が更新される様子を示す。

    \begin{figure}[H]
        \centering
        \begin{tabular}{c}
        \begin{minipage}{0.45\hsize}
        \centering
        \includegraphics[clip, width=1.0\columnwidth]{./figure/LOCO/fit_loco_sH.png}
        \hspace{16mm} [1]水平方向
        \end{minipage}
        \begin{minipage}{0.45\hsize}
        \centering
        \includegraphics[clip, width=1.0\columnwidth]{./figure/LOCO/fit_loco_sV.png}
        \hspace{16mm} [2]垂直方向
        \end{minipage}
        \end{tabular}
        \caption{4極スキュー成分のフィッティング.}
        \label{loco_skew}
    \end{figure}

以上の操作によって、応答行列の誤差は水平方向で5.2 \si{\mu m}、垂直方向で2.8 \si{\mu m}と収束した。これはシミュレーションに導入するのに十分な精度である。更に誤差を小さくするためにはゲインの測定とクランチの効果を計算すればよい。実験結果であるLOCOモデルと計算モデルオプティックスの比較を図\ref{loco_optics}に示す。また、ウィグラーの電源を切り替えた時の結果を図\ref{loco_wiggler}に示す。

    \begin{figure}[H]
        \centering
        \includegraphics[width=120mm]{./figure/LOCO/optics.png}
        \caption{PFリングのオプティックス。実線はデザインモデルのベータトロン関数を表す。破線は挿入光源の影響によって歪んだベータトロン関数を表す。ベータトロン関数の歪みはアクセプタンスにあらわれる。入射シミュレーションでは実際の運転時にビームが受けるベータトロン関数を適用した。}
        \label{loco_optics}
    \end{figure}

    \begin{figure}[H]
        \centering
        \includegraphics[width=120mm]{./figure/LOCO/wiggler.png}
        \caption{PFリングのオプティックス(wiggler on/off from LOCO model)。点線は挿入光源の電源を入れた時のベータトロン関数を表す。破線は挿入光源の電源を入れていないときのベータトロン関数を表す。}
        \label{loco_wiggler}
    \end{figure}

%******************************************************************************
\section{入射率の見積もり}
現実の加速器要素は非線形成分を含むため軌道の運動方程式は解析的に解けない。そのため、粒子の追跡では加速器要素を無限小区間に分割して正準変換のヤコビアンを転送行列に対応させる。この場合、入射ビームに対応する分布を持った各粒子は、初期値毎にその周辺で運動方程式を線形化して粒子の軌道が求められる。この粒子追跡を放射減衰のスケールまで行い入射率を評価した\cite{takagi}。

入射シミュレーションはLOCOモデルと設計モデルのオプティックスを使用した。LOCOモデルのオプティックスはビーム測定したものである。表\ref{Aperture_phy}にPFリングのフィジカルアパーチャーとベータトロン関数をまとめる。アクセプタンスは水平方向に超電導ウィグラー (ID14)とセプタム壁を、垂直方向にアンジュレータ (ID16)を設定した。また、ID14とセプタムに付随するアブゾーバーを設定した。LOCOモデルで予測した水平β関数を使って、規格化振幅はセプタム出口で6.50から5.84、ID14で5.67から6.45となり、アパーチャーはセプタム出口で最も狭くなっていることがわかる。

    \begin{table}[H]
    \caption{フィジカルアパーチャーとベータ関数}
    \label{Aperture_phy}
    \centering
    \begin{tabular}{ c c c c c } \hline
    Position & \multicolumn{2}{c}{Aperture[mm]} & \multicolumn{2}{c}{Betatron function[m]} \\ \cline{2-5}
     & Horizontal & Vertical & Design & LOCO \\ \hline
    ID16 & - & 7.5 & 5.00 & 5.07 \\
    ID14 & 16 & - & 7.96 & 6.15 \\
    SEPTUM & 20 & - & 9.44 & 11.71 \\ \hline
    \end{tabular}
    \end{table}

入射率の計算は以下の条件で行う。キッカー電磁石のタイミングはピーク位置を起点としたシングルキックとエネルギー偏差を+1\%加えたシンクロトロン入射を考え、オプティックスは設計モデルを使用する。またK3の蹴り角は1.51 \si{mrad}、K4の蹴り角は3.52 \si{mrad}とした。入射ビームに対応する初期粒子は実験結果の位相空間情報でガウス分布を生成した。生成粒子数は1000、周回数を1000として粒子追跡の最中に粒子の座標が表\ref{Aperture_phy}のフィジカルアパーチャーを超えた場合、ビームは失われるとする。この時、周回毎に変化する粒子数をプロットした結果を図\ref{eff_inj0}に示す。縦軸は粒子数を示し、横軸はリングの周回数を表す。

    \begin{figure}[H]
        \centering
        \includegraphics[width=120mm]{./figure/05_Simulation/eff_inj0.png}
        \caption{生存粒子数の変化}
        \label{eff_inj0}
    \end{figure}

    \begin{table}[H]
    \caption{入射ビームの位相空間情報}
    \label{parameter_Inj}
    \centering
    \begin{tabular}{ c c c } \hline
     & Position[mm] & Angle[mrad] \\ \hline
    Horizontal & 30.5$\pm$3.20 & 3.11$\pm$0.42 \\
    Vertical & 0.06$\pm$0.23 & 1.02$\pm$0.23 \\ \hline
    \end{tabular}
    \end{table}

図\ref{eff_inj0}の凡例originalは表\ref{parameter_Inj}の位相空間上を用いて入射したもので効率は55.0\%である。凡例Fitthe vertical directionはビーム輸送路の補正電磁石 (CM)で垂直方向の位相空間を原点に調整した条件で効率は60.4\%に改善される、凡例Fit the septum angleはセプタム電磁石の調整した条件で効率は69.6\%となった。ベータトロン振動の影響による粒子数の変化を追いかけるには図\ref{eff_inj0}から、ベータトロン振動による影響をみる100ターンで十分である。トラッキングの周回数を100ターンに減らして下流のキッカー電磁石の蹴り角の組み合わせを変えながらキッカージャンプ入射による入射率の変化を計算した(図\ref{Inj_275_305})。また入射ビームの水平位置が30.5 mmと27.5 mmの場合を比較する。

    \begin{figure}[htbp]
        \centering
        \begin{tabular}{c}
        \begin{minipage}{0.45\hsize}
        \centering
        \includegraphics[clip, width=1.0\columnwidth]{./figure/05_Simulation/Pic_Inj_275.png}
        \hspace{16mm} [1] x=27.5 \si{mm}
        \end{minipage}
        \begin{minipage}{0.45\hsize}
        \centering
        \includegraphics[clip, width=1.0\columnwidth]{./figure/05_Simulation/Pic_Inj_305.png}
        \hspace{16mm} [2] x=30.5 \si{mm}
        \end{minipage}
        \end{tabular}
        \caption{設計モデルを使った入射率(シングルターンキック入射)}
        \label{Inj_275_305}
    \end{figure}

図\ref{Inj_275_305}のカラーバーの範囲は入射率(生存粒子数/生成粒子数=1000)を表し、シングルターン入射時のキッカージャンプ入射によるキッカーバンプの振動を犠牲にしたビームロスの緩和が評価される。また入射ビームの水平位置が遠ざかると入射率が落ち込むことが図\ref{Inj_275_305}の比較よりわかる。

同様の計算をLOCOモデルのオプティクスを使用して計算を行い、またマルチターンキックの場合と比較した(図\ref{Inj_LOCO})。LOCOモデルを使用した場合、図\ref{Inj_LOCO}の左図は図\ref{Inj_275_305}の右図と比較して、入射率は設計モデルの時と比べて落ち込むことがわかる。これは、表\ref{parameter_Inj}のSEPTUMでアパチャーが狭まる影響である。図\ref{Inj_LOCO}の右図、1ターン目と2ターン目のマルチターン入射では、入射ビームの振幅を小さくするため、入射率が改善される様子がわかる。

    \begin{figure}[H]
        \centering
        \begin{tabular}{c}
        \begin{minipage}{0.45\hsize}
        \centering
        \includegraphics[clip, width=1.0\columnwidth]{./figure/05_Simulation/Pic_Inj_305_LOCO.png}
        \hspace{16mm} [1] シングルターン入射
        \end{minipage}
        \begin{minipage}{0.45\hsize}
        \centering
        \includegraphics[clip, width=1.0\columnwidth]{./figure/05_Simulation/Pic_Inj_305_LOCO_mul.png}
        \hspace{16mm} [2] 1から2ターン目のマルチターン入射
        \end{minipage}
        \end{tabular}
        \caption{LOCOモデルを使った入射率}
        \label{Inj_LOCO}
    \end{figure}

前節の計算結果とビームスタディの比較からマルチターンキック入射は0ターンから3たん目までのシミュレーションで再現できることを確認した。入射率の計算を0から3ターンまでの影響に拡張して行う。これを以下の図\ref{injeff_03}に示す。

    \begin{figure}[H]
        \centering
        \includegraphics[width=120mm]{./figure/06_Optimization/injeff_mul03.png}
        \caption{LOCOモデルの入射率(0-3 turn)}
        \label{injeff_03}
    \end{figure}

図\ref{injeff_03}から最適な調整を実施しても入射率は7割程度と改善されないことがわかった。またマルチターンキックは蓄積ビームを振動させる原因となるが、入射率に対しては条件の許容を緩和することがわかった。入射に関する課題を以下にまとめる。
\begin{itemize}
\item キッカーバンプマッチング調整

キッカーバンプマッチングの計算より、タイミングをピーク位置で揃えるより、それぞれのタイミングを変えることで振動を抑制できることがわかった。
\item 入射ビームの垂直振動の抑制、セプタム電磁石の蹴り角の補正

入射ビームの水平位置が離れているため、垂直振動を含む入射の効率は落ち込むこと、セプタムの蹴り角が足りていないことがわかった。キッカー入射の要件からも垂直方向の位相空間情報は原点にあることが望まれる。またセプタム電磁石はすでに設計の最大値付近で運転しているため強めることは難しく、セプタム電磁石をリングに寄せる必要がある。
\end{itemize}

%*******************************************************************************
\newpage
\chapter{入射最適化}
本章では、マルチターンキックのおけるキッカーバンプの蓄積ビーム振動抑制を目的に実施したビームベース最適化 (Beam based optimization:BBO)を記述する。入射パラメータの測定は、実際のモデルとデザインモデルの応答の違いを測定することによって補正を行う、ビームベース補正 (Beam based correction:BBC)にあたる。加速器は多くの誤差を含むモデルであり、検討していない要因の誤差が残る。ビームベース最適化では、ビームの応答をみて調整するので、誤差の要因を含めて最適解に収束される。
%*******************************************************************************
\section{最適化方法}
加速器は多くの部品で構成される複雑なシステムである。加速器は恒常的にアライメントエラーや磁場誤差、挿入光源、フリンジフィールドなど多くの誤差を含む。また、キッカー入射の場合、バンプ内に含まれる非線形磁場や渦電流の効果、電磁石の個体差など考慮すべき誤差が追加される。これらの誤差の影響を詳細に調査し補正することは困難である。そのため、ビームベース最適化 (BBO)が採用される。ビームベース最適化とは、評価関数を設定しパラメータの調整で最適化を行う手法である。適切な評価関数とパラメータの設定がされた場合、パラメータ間の相互作用は無視できる利点をもつ。しかしノイズに対する堅牢性をもつアルゴリズムが求められる。

加速器パラメータの調整は手動と自動の2種類がある。手動の場合、1つのパラメータを複数回の反復を行って目標関数の変化を調べる。これは多パラメータの1次元スキャンに相当する。小規模の問題は手動による方法で十分である。しかし、パラメータの多い問題や非線形問題は反復回数が増大し、収束までの効率は悪化する。自動の場合、一次勾配情報を用いた複数のパラメータを同時に調整する方法\cite{RCDS}と確率的アルゴリズムを用いた複数のパラメータをランダムに最適化する方法\cite{GA_method}の2種類が主に用いられる。加速器施設における自動調整は目的に合わせて必要な調整方法が各施設で開発されてきた。線形収束力が主なリングかつ振動抑制が目的の本研究では勾配法による調整で最適化を目指す。

本研究の最適化は、マルチターンキック状況下の蓄積ビームの振動抑制を目標にビームベース最適化を実施する。ここでは蓄積ビームの振動抑制の精密調整と入射率を目的とした入射調整は取り扱わない。精密な蓄積ビームの振動抑制を検討する場合には、条件を交換して最適化を反復すればよい。また入射率改善を検討する場合には、入射ビームの位相空間情報を調整して入射率を目標関数にすればよい。これらの調整は、最適化が可能であることを確認したうえで、セプタムの交換後に実施する予定である。

キッカーバンプの調整では自動調整を採用する。自動調整のアルゴリズムは最急降下法と直線探索法を組み合わた単純なものとした。キッカーバンプの調整は、上流のキッカーパラメータを固定とし、下流のキッカーパラメータをフリーとした。PFのキッカーで調整可能なパラメータは蹴り角とパルスタイミングの2種類である。アルゴリズムにはBPM精度近くでのノイズを考慮した調整や収束速度を早くするための工夫は次の段階で加える予定である。

最急降下法と直線探索について説明する。最急降下法は、
\begin{equation}
minimize f(x)
\end{equation}
を解くことに相当する\cite{conj}。ここで$f(x)$は式(\ref{objective_function})の目標関数である。初期点$x_0$から調整をはじめて、
\begin{equation}
x_{k+1} = x_k + \alpha_k d_k
\end{equation}
の更新を重ねる。ここで$\alpha_k$と$d_k$はステップ幅と探索方向である。探索方向は目標関数を微分して得られる。探索方向が得られるとステップ幅を調整する、直線探索がなされる。直線探索ではスッテプ幅を足しながら前後の目標関数を比較して小さくなる場合に交換した。
目標関数$f(x)$はキッカーバンプマッチングで導入した誤差キックによる蓄積ビームの振動量に相当する、
\begin{equation}
\label{meas_function}
f(x)_{calc} = \sum_{k}^{iteration} \sqrt{ \frac{1}{N} \sum_{j}^{turn} \sum_{i}^{BPM} x_i^2 }
\end{equation}
とした。ここで、iはBPMの数、jはターンごとの振動の測定範囲、kは測定の回数を表す。

%*********************************************************************
\section{収束過程}
測定方法を説明する。目標関数の測定に必要なBPMは運転時に設置されていたLiberaの6台使用した。表\ref{meas_opt}に使用したBPMの名前と設置場所をまとめる。測定の開始タイミングは、入射トリガーがきてキッカー電磁石が励磁された後、最初の数十ターンを除いたタイミングを起点として、測定範囲を800ターン分取得した。そのときの測定範囲を図\ref{opt_range}に示す。測定回数は5回とした。

    \begin{figure}[H]
        \centering
        \includegraphics[width=110mm]{./figure/06_Optimization/meas_range.png}
        \caption{測定範囲(pflibera07)}
        \label{opt_range}
    \end{figure}

    \begin{table}[H]
    \caption{BPMと設置場所}
    \label{meas_opt}
    \centering
    \begin{tabular}{ c r } \hline
    Name & Position \\ \hline
    pflibera02 & BPM023 \\
    pflibera03 & BPM273 \\
    pflibera04 & $U16_{down}$ \\
    pflibera05 & $U16_{middle}$ \\
    pflibera07 & $U16_{up}$ \\
    pflibera09 & BPM022 \\ \hline
    \end{tabular}
    \end{table}

初期点$x_0$は運転当時のパラメータを基本とした。K1の蹴り角は、アブソーバーで削れない限界近くであるためバンプハイトの許容量を確保して小さくした。測定当時のキッカーパラメータを表\ref{optstudy_kicker}にまとめる。

    \begin{table}[htbp]
    \caption{測定時のキッカーパラメータ}
    \label{optstudy_kicker}
    \centering
    \begin{tabular}{ c  c c } \hline
     & Angle [mrad] & Delay time [nsec] \\ \hline
    $K1$   & 2.02 & 1890 \\
    $K2$   & 0.00 & 1800 \\
    $K3$   & 1.51 & 1720 \\
    $K4$   & 3.47 & 2400 \\ \hline
    \end{tabular}
    \end{table}

以上の条件で目標関数の測定を行い、シングルバンチとマルチバンチでキッカーバンプマッチングの調整をした。探索方向$d_k$は、フリーパラメータを$\pm 0.5\%$変化させて、フィッティングした一次関数の傾きとした。ここでのフリーパラメータはK3、K4のディレイと蹴り角である。反復はディレイと蹴り角を交互に変えて行い、収束すると終了した。

キッカーバンプマッチング調整は10 \si{mA}のシングルバンチとマルチバンチの蓄積ビームを使用する。シングルバンチの蓄積ビームの振動量はキッカーのパルス波形のピーク位置で蹴られた値に相当する。マルチバンチでは、設定前後を含むバンチの振動を平均した結果に相当する。最適化の過程ではパルスタイミングと蹴り角を交互に調整しながら、収束に従ってスッテプ幅は小さくした。図\ref{BBO}に直線探索の反復を含めた目標値の変化を示す。また、反復によって得られたキッカーパラメータを表\ref{BBO_kicker}にまとめる。

    \begin{figure}[H]
        \centering
        \begin{tabular}{c}
        \begin{minipage}{0.45\hsize}
        \centering
        \includegraphics[clip, width=1.0\columnwidth]{./figure/06_Optimization/single_ite.png}
        \hspace{16mm} [1] シングルバンチ
        \end{minipage}
        \begin{minipage}{0.45\hsize}
        \centering
        \includegraphics[clip, width=1.0\columnwidth]{./figure/06_Optimization/mul_ite.png}
        \hspace{16mm} [2] マルチバンチ
        \end{minipage}
        \end{tabular}
        \caption{目標値の変化}
        \label{BBO}
    \end{figure}

    \begin{table}[H]
    \caption{キッカーパラメータ}
    \label{BBO_kicker}
    \centering
    \begin{tabular}{ c c c } \hline
    Name & Angle [mrad] & Delay time [nsec] \\ \hline
    single bunch & & \\
    $K1$   & 2.02 & 1890 \\
    $K2$   & 0.00 & 1800 \\
    $K3$   & 1.51 & 1765 \\
    $K4$   & 3.43 & 2392 \\
    multi bunch & & \\
    $K1$   & 2.02 & 1890 \\
    $K2$   & 0.00 & 1800 \\
    $K3$   & 1.50 & 1795 \\
    $K4$   & 3.57 & 2388 \\ \hline
    \end{tabular}
    \end{table}

\section{最適化結果}
\subsection*{振動量の比較}
図\ref{bbo_ori_single}と図\ref{bbo_opt_single}に最適化前後のモニター(pflibera07)で観測したシングルバンチ時の蓄積ビームの振動を示す。横軸のSamplesは時間を表し、ターンごとの間隔(=624 \si{nsec})に相当する。縦軸はビーム位置を表し、間隔は100 \si{\mu m}の振動に相当する。
    \begin{figure}[H]
        \centering
        \includegraphics[width=120mm]{./figure/06_Optimization/ori_single_bunch.png}
        \caption{最適化前の蓄積ビームの振動 (シングルバンチ)}
        \label{bbo_ori_single}
    \end{figure}

    \begin{figure}[H]
        \centering
        \includegraphics[width=120mm]{./figure/06_Optimization/opt_single_bunch.png}
        \caption{最適化後の蓄積ビームの振動 (シングルバンチ)}
        \label{bbo_opt_single}
    \end{figure}

最適化による結果、蓄積ビームの振動量は抑制されていることが確認できる。このときの振動はK1の蹴り角がピーク位置を中心に固定したときの振動で最適化されている。次に最適化後の蓄積ビームの振動量を、全てキッカーのディレイを変えながら調べた。図\ref{opt_result_sin}にキッカーのディレイを0から0.7 \si{msec}まで8 \si{nsec}刻みで遅延時間方向に掃引して、目標関数を測定した結果を示す。

    \begin{figure}[H]
        \centering
        \includegraphics[width=110mm]{./figure/06_Optimization/single_obj.png}
        \caption{目標値の周期構造}
        \label{opt_result_sin}
    \end{figure}

図\ref{opt_result_sin}は目標値の変化が周期構造となることを示す。最適化されたパラメータではタイミングを調整してパルス幅の差異や反射部分を含めた蓄積ビームの振動抑制が達成されている。最適化前のパラメータはキッカーの励磁タイミングは蹴り角がピーク位置となるように揃えられており、パルス幅や反射で蹴り角の誤差が大きくなるタイミングで振動量の変化が大きいことを確認できる。

同様に図\ref{bbo_ori_multi}と図\ref{bbo_opt_multi}に最適化前後のモニター(pflibera07)で観測したマルチバンチ時の蓄積ビームの振動を示す。横軸のSamplesは時間を表し、ターンごとの間隔(=624 \si{nsec})に相当する。縦軸はビーム位置を表し、間隔は100 \si{\mu m}の振動に相当する。

\newpage
    \begin{figure}[H]
        \centering
        \includegraphics[width=120mm]{./figure/06_Optimization/ori_multi_bunch.png}
        \caption{最適化前の蓄積ビームの振動 (マルチバンチ)}
        \label{bbo_ori_multi}
    \end{figure}

    \begin{figure}[H]
        \centering
        \includegraphics[width=120mm]{./figure/06_Optimization/opt_multi_bunch.png}
        \caption{最適化後の蓄積ビームの振動 (マルチバンチ)}
        \label{bbo_opt_multi}
    \end{figure}

最適化による結果、蓄積ビームの振動量は抑制されていることが確認できる。マルチバンチ時の振動は設定値前後の蹴り角でもバンチは蹴られビームの振動は平均化された値が返される。次に全てキッカーのディレイを変えながら最適化後の蓄積ビームの振動量の変化を調べた。図\ref{opt_result_sin}にキッカーのディレイを0から0.7 \si{msec}まで20 \si{nsec}刻みで遅延時間方向に掃引して、目標関数を測定した結果を示す。

    \begin{figure}[H]
        \centering
        \includegraphics[width=110mm]{./figure/06_Optimization/mul_obj.png}
        \caption{目標値の周期構造}
        \label{Opt_multi_result}
    \end{figure}

図\ref{Opt_multi_result}より目標値は周期構造を持つことがわかる。またシングルバンチの図\ref{opt_result_sin}と比較して蓄積ビームの変化量は平均化されていることがわかる。最適化されたパラメータは目標値が最小化されていることを確認した。以上の操作は、全バケットの振幅を最小限に抑えるために機能する。

\subsection*{シミュレーションとの比較}
前節の比較によって、最適化から蓄積ビームの振動量は抑制されたことを確認した。ここでは、入射条件の計算の章にて調べた、キッカーバンプマッチングの計算を最適化前後のキッカーパラメータを使って行う。図\ref{BBO_com}にK3とK4のディレイを設定値からずらした時の蓄積ビームの振動量をカラープロットした結果を示す。横軸と縦軸のキッカーのタイミングは設定値を原点とした値を表す。表カラーバーの範囲は、リングに設置されている全てのBPM設置場所で観測される蓄積ビームの振動量をマルチターンキックが終わってから2周期分積算した値を表す。

    \begin{figure}[H]
        \centering
        \begin{tabular}{c}
        \begin{minipage}{0.45\hsize}
        \centering
        \includegraphics[clip, width=1.0\columnwidth]{./figure/06_Optimization/com_ori.png}
        \hspace{16mm} [1] 元の設定値
        \end{minipage}
        \begin{minipage}{0.45\hsize}
        \centering
        \includegraphics[clip, width=1.0\columnwidth]{./figure/06_Optimization/com_opt.png}
        \hspace{16mm} [2] ビームベース最適化後
        \end{minipage}
        \end{tabular}
        \caption{最適化前後の目標値のマッピングのシミュレーション結果}
        \label{BBO_com}
    \end{figure}

比較の結果、図\ref{BBO_com}の右図よりタイミング調整は最小値付近まで収束していることがわかった。蹴り角調整も同時に行ったが殆ど変化はみられていない。これは前章のシミュレーションの通り、タイミングをピーク位置で揃えるよりそれぞれのタイミングを変えることで振動を抑制できることを表す。蹴り角調整に関しては、予めシミュレーションでキッカージャンプ入射による入射率を考慮した蹴り角の組み合わせやバンプを閉じるためにフィッティングした組み合わせなど目的に合わせて計算を行い、これを始点とする準備が必要であることがわかった。

蓄積ビームの振動に関する課題を以下にまとめる。

\begin{itemize}
\item 振動測定のノイズ処理

ビームベース最適化による結果、蓄積ビームの振動は\si{ \pm 100\mu m}の範囲に抑制されている。この振動量はBPMの測定精度と同程度である。測定精度を良くするため、バンチの電荷量を増やすことや測定回数を増やすこと、測定回路を見直すなど必要になる。
\item フィルパターンの処理

マルチバンチの振動測定はフルフィルではないため、測定タイミングとバケットの位置の依存性が含まれる。現在のBPMの測定精度では、バケット依存性を考慮した最適化まで扱えないため問題とならないが、測定精度の改善後は考慮しないといけない。
\end{itemize}

%*******************************************************************************
\newpage
\chapter{結論}
本研究は、PFリングにおいて問題であった入射率の低下の問題を、以下の手順から改善を目指した。はじめにビームベース測定で入射パラメータの測定と校正を行った。次に得られた入射パラメータを用いて入射シミュレーションを行い問題の原因を調べた。最後に問題を改善するための条件の探索をシミュレーションとビームスタディの両方から実施した。結果をまとめ、今後の課題を述べる。

\subsection*{入射パラメータの測定}
ビームベース測定で調べた要素は、入射点での蓄積ビームとセプタム壁、入射ビームの位置関係、キッカー電磁石のパルス波形、セプタム電磁石の励磁曲線である。入射点でのビーム位置関係の測定と同様の方法でビーム輸送路終端部の補正電磁石の応答も測定した。しかし、入射ビームのばらつきとビームダクトの端を通ることによる電極からの非線形応答が原因で測定できていない。

    \begin{table}[H]
    \caption{入射点での入射ビームの位相空間情報}
    \label{sum_1}
    \centering
    \begin{tabular}{ l c c } \hline
    & Position [mm] & Angle [mrad] \\ \hline
    calculation & & \\
    Horizontal & $30.5$ & $3.1$ \\ 
    Vertical    & $0$     & $0$   \\ \hline
    measurement & & \\
    Horizontal & $27.0$ & $2.0$ \\ 
    Vertical    & $0.06$ & $1.02$ \\ \hline
    \end{tabular}
    \end{table}

入射点でのビームの位置関係は表\ref{sum_1}の結果が得られた。測定結果は、水平方向で入射ビームの位置が遠くなり蹴り角が足りていないことで、入射振動が大きくなっていることがわかった。また垂直方向で角度をもち入射されていることがわかった。これはPFの事情として入射器と蓄積リングの高さが違うため、戻されていないことに由来する。入射振幅が大きくなり、許容量が不十分な状況では垂直方向の振動も入射率の低下として現れる原因となる。

キッカー電磁石の測定ではパルス波形を得た。この情報からピーク位置をみた励磁特性と波形から得るマルチターンの蹴り角を調べた。励磁特性の結果から制御系の設定とビームベースの結果を比較して校正されている。またキッカー電磁石のパルス幅と反射の影響を確認した。

セプタム電磁石の測定では励磁曲線を調べた。励磁曲線の測定前にパルス波形のピーク位置で入射ビームを蹴れていること、セプタム電磁石の最大磁場強度で飽和していないことを確認している。測定結果からビーム輸送路終端部のセプタム電磁石で設定より大きく蹴っていることがわかった。入射調整で設定された蹴り角は、現在の入射ビームの位置に対して整合性をもつ。セプタム電磁石の応答は、入射ビームを用いた測定になるため、振幅の大きさから測定条件が限られる。

\subsection*{入射条件の計算}
入射条件の計算では、入射パラメータの測定結果を導入してキッカーバンプと入射を計算した。キッカーバンプの計算ではマルチターンキックでバンプ形成するために必要な蹴り角とマッチングがずれたときの振動の大きさについて述べた。入射の計算では入射振動の抑制に必要な調整とマルチターンキック時の入射率について述べた。

マルチターンキックの蹴り回数を増やすとバンプを閉じる時に計算した結果は、途中で振幅が大きくなり使うことはできない。そのためマッチングがずれた時の振動の大きさを目標値にして、バケット全体で評価した値を導入した。マルチターンキックの数を増やして実際のビームの応答に合っているのかを確認する必要がある。入射振動の抑制は入射ビームの調整を、水平方向の位置では入射ビームのばらつきを考慮しセプタム位置に近づけて、角度は正規化位相空間で原点をとるようにし、垂直方向の位相空間情報を原点にとり達成される。入射率の計算では、入射パラメータの測定で得られた入射ビームの位相空間情報とマルチターンキックを導入してキッカージャンプによる効率改善の効果を調べた。

\subsection*{入射調整}
入射条件の探索では、はじめに入射条件の計算で導入したキッカーバンプマッチングを評価した目標値とビームスタディの結果を比較して目標関数の正当性を評価した。次に入射率の計算を拡張した。最後にキッカーバンプマッチングの調整を実施した。

目標関数の評価では、実際のビームに含まれる非線形効果による違いを持つが一致していることを確認した。この評価によって、マルチターンキックの主要な部分はシミュレーションに反映できたとわかる。入射率の計算では、測定したオプティックス、位相空間情報、マルチターンキックなど全ての結果を導入して再計算を行った。その結果、入射ビームの耐性が落ちた状況でも入射できたのはマルチターンキックによる効果だとわかった。しかし入射調整をしてキッカージャンプを含めても、効率は7割までしか改善できないことがわかった。キッカーバンプマッチングの調整では、PFリングのキッカー電磁石で調整可能なディレイタイムと蹴り角を調整してマッチングを行った。これは最急降下法を用いて目標関数を最小化することに相当する。調整の結果、蓄積ビームの振幅抑制に成功した。

\subsection*{今後の課題}
入射に関するパラメータの測定と計算、調整のプロセスを実施した。入射率は調整による改善も限界があり、現在のセプタムが遠い状況では許容量は低いため取り扱っていない。セプタム電磁石が交換された後は、入射ビームの調整を行い、リングより上流の問題を取り扱う。

%*****************************************************************
\newpage
\chapter*{謝辞}
\addcontentsline{toc}{chapter}{\numberline{}謝辞}
はじめに本研究の機会を与えてくださった、高エネルギー加速器研究機構の小林幸則先生と広島大学の島田賢也先生に感謝を表します。

本論文は、高エネルギー加速器研究機構の放射光科学研究施設で行いました、PFリングの入射スタディをまとめたものです。本研究と論文の執筆に当たって、多くの方々にお世話になりました。この場を借りしてお世話になった皆様にお礼申し上げます。

広島大学の松葉俊哉助教には、学部の頃からご指導頂きました。KEKの原田健太郎准教授には、研究から論文の執筆、進路関係の相談まで幅広く、気を配っていただきました。加速器科学の基礎から入射のことまで、研究から培われた経験と知識を学べたことは非常に有益でした。帯名崇教授には、全ての入射スタディに付き合って頂きました。計算機科学の豊富な知識とEpicsについてご教授して頂きました。東直特別助教には、発表練習から申請書の書き方までお世話になりました。共同研究者の長橋進也技師、上田明専門技師、高井良太准教授、高木宏之准教授には、実験準備から必要なデータの準備、成果報告まで、入射スタディの全てに関して協力頂きながら研究活動に付き合っていただきました。加速器七系の方々には、光源打ち合わせや交流会の場で報告内容の議論や学会、修士論文の練習に付き合っていただきました。非常に有意義な研究生活を過ごすことができました。改めて本研究にご協力いただいた皆様に心より感謝いたします。

最後に、これまで学生生活を支援いただいた私の家族に感謝を表します。
%*****************************************************************
\appendix
\newpage
\chapter{入射方式}
ここでは入射振幅を抑制する目的で、キッカー入射と組み合わせて用いられる、横方向の振動を犠牲にしたキッカージャンプ入射と縦方向の振動を犠牲にしたシンクロトロン入射を議論する。
\section{キッカージャンプ入射}
通常、キッカーバンプは位相空間上で原点と戻るように調整されるが入射ビームの振幅の増大が問題となり、アパーチャー内に収められないとき、下流のキッカーK3、K4を大きく蹴る。これをキッカージャンプ入射と呼ぶ\cite{kicker_jump}。蓄積ビームの振動はバンプを崩すことになるので蓄積ビームの振動は励起されるが、入射の要件であるキッカーバンプマッチングと入射率の両立から、以上の操作が要求される場合がある。キッカージャンプ時の入射ビームと蓄積ビームの軌跡を以下の図\ref{accep_jump}に示す。

     \begin{figure}[H]
        \centering
        \includegraphics[width=100mm]{./figure/03_Theory_Injection/accep_jump.png}
        \caption{キッカージャンプによる入射ビームと蓄積ビームの軌跡}
        \label{accep_jump}
    \end{figure}

キッカージャンプによる入射振動抑制効果は入射率として測定される。計算では粒子の初期分布を考えた各粒子に対する軌道の運動方程式を解くことで評価できる。ただし、キッカージャンプを使うと蓄積ビームが大きく振動し、ユーザーの実験ができなくなる。

%*****************************************************************
\section{シンクロトロン入射}
シンクロトロン入射は縦方向入射(Longitudinal Injection)と共にビームの進行方向に対する入射を指す。ビームの進行方向、つまり縦方向に対する位相空間情報はRFの位相に対するビームのタイミングに相当する。縦方向入射を説明するため、はじめに位相空間安定性について説明する。位相空間安定性とはRFの加速電圧と制動放射の釣り合う位相を中心にビームが集群する効果を指す。この原理を図\ref{phase_stability}で説明する。

     \begin{figure}[H]
        \centering
        \includegraphics[width=120mm]{./figure/03_Theory_Injection/phase_stability.png}
        \caption{位相空間安定性}
        \label{phase_stability}
    \end{figure}

図\ref{phase_stability}は正弦的に変化するRFの電圧と設計電圧$V_a$を示す。ここで粒子がRFの加速電圧からずれて到着した場合を考える。到着するタイミングのずれは軌道長の変化を表し、エネルギーの高い粒子が入射されると質量が増加して、偏向電磁石の磁場の影響を受けにくくなる (Magnet Rigidity)。すると軌道長は増加してタイミングのずれは遅れるタイミングとなる。遅れたタイミングの粒子は設計電圧より大きい電圧で加速され設計粒子の軌道に近づく(赤矢印)。またエネルギーの低い粒子が粒子が入射された場合も、軌道長は減少してタイミングは早くなり設定電圧より低い電圧で加速されるため設計軌道に近づく(青矢印)。以上が集群の原理である。

位相空間安定性の原理によって縦方向にずれた粒子は集群されることを上記で示した。縦方向の振動はシンクロトロン振動と呼ぶ。つぎに縦方向のずれ、つまりエネルギーと時間のずれに対する許容範囲を説明する。エネルギーと時間のずれに対する許容範囲をセパラトリクス呼ぶ。セパラトリクスを図\ref{separatrix}に示す。

     \begin{figure}[H]
        \centering
        \includegraphics[width=120mm]{./figure/03_Theory_Injection/separatrix.png}
        \caption{セパラトリクス($\phi=\pi/2$)}
        \label{separatrix}
    \end{figure}

図\ref{separatrix}は正弦波の位相を$\pi/2$としたときのセパラトリクスを示す。設計電圧の付近で電圧の傾きは線形で集束力を線形と考えた場合、シンクロトロン振動の軌跡は円となる。この円の範囲をセパラトリクスと呼ぶ。最後に実際の加速器でのセパラトリクスを考える。図\ref{separatrix_SLS}はSwiss Light Source (SLS)で計算されたシンクロトロン振動である\cite{injection_SLS}。

     \begin{figure}[H]
        \centering
        \includegraphics[width=100mm]{./figure/03_Theory_Injection/separatrix_SLS.png}
        \caption{シンクロトロン入射(出典\cite{injection_SLS})}
        \label{separatrix_SLS}
    \end{figure}

図\ref{separatrix_SLS}は、ブースターシンクロトロンから渡される1000個それぞれの粒子の運動方程式を解いて、15000回転の軌跡をプロットしたものである。エネルギーと時間にずれをもって入射された粒子はシンクロトロン放射の効果で減衰して設計軌道に捕獲されていることがわかる。以上より、シンクロトロン入射とは、蓄積ビームから時間とエネルギーの片方、もしくは両方をずらすことで、縦方向位相空間で分離を行い、ビームを入射すること操作を示す。

PFリングではキッカー入射にエネルギー偏差を加えた入射を行っている。入射ではLINACから渡される粒子に$\Delta E/E=0.01\% $のエネルギー偏差を加えている。その時の入射ビームの軌跡を図\ref{inj_syn_pf}に示す。図\ref{inj_syn_pf}のエネルギー偏差を加えたときの振幅は加えていないときと比較して小さくなることが確認できる。これは正の分散がある入射部でエネルギー偏差を加えているため横方向の振幅が縦方向に交換されることを表す。

     \begin{figure}[H]
        \centering
        \includegraphics[width=120mm]{./figure/03_Theory_Injection/separatrix_KEK.png}
        \caption{シンクロトロン入射の入射ビームの軌跡}
        \label{inj_syn_pf}
    \end{figure}

図\ref{inj_syn_pf}では横方向の振幅が縦方向に交換されることを表した、その効果を示すためクーランシュナイダー不変量Wを縦軸に、リング一周の周長を横軸にとった計算結果を図\ref{inj_syn_courant_pf}に示す。図\ref{inj_syn_courant_pf}のBetatronは設計振幅から一次の分散の振幅を引いたクーランシュナイダー不変量であり、DispersionはBetatronに一次の分散の振幅分を加えた値を示す\cite{harada_pri}。

     \begin{figure}[H]
        \centering
        \includegraphics[width=120mm]{./figure/03_Theory_Injection/separatrix_KEK2.png}
        \caption{シンクロトロン入射のクーランシュナイダー不変量}
        \label{inj_syn_courant_pf}
    \end{figure}

図\ref{inj_syn_courant_pf}の最初の振幅は入射点から下流のキッカー電磁石2台で小さくされた後、分散のある部分でベータトロン振動の周りに偏差を加えていることがわかる。この偏差が分散の振動成分である。入射点 (Length=0 m)での、Betatronの振動成分を取り出した値と元の値の差分が、交換された量に相当する。



%*****************************************************************
\newpage
\addcontentsline{toc}{chapter}{\numberline{}参考文献}
\renewcommand{\bibname}{参考文献}
\begin{thebibliography}{99}
% Chapter 1
% Introduction
\bibitem{phase_focusing}
EDEIN M. McMillan,
\newblock ``THE SYNCHROTRON-A PROPOSED HIGH ENERGY PARTICLE ACCELERATOR",
\newblock Phys. Rev. 68, 143, 1945.

\bibitem{Courant}
E. Courant, M.S. Livingston and H. Snyder,
\newblock ``THE STRONG-FOCUSING SYNCHROTON-A NEW HIGH ENERGY ACCELERATOR",
\newblock Phys. Rev. 88, 1190, 1952.

\bibitem{radiation}
F.R. Elder, A.M. Gurewitsch, R. V. Langmuir and H. C. Pollock.
\newblock ``RADIATION FROM ELECTRONS IN A SYNCHROTRON",
\newblock Phys. Rev. 71, 829, 1947.

\bibitem{PF}
K. Harada, Y. Kobayashi, T. Obina, A. Ueda and M. Izawa,
\newblock ``LOW EMITTANCE OPTICS AT THE PHOTON FACTORY",
\newblock Proceedings of the 2003 Particle Accelerator Conference, pp. 3201-3203, 2003.

\bibitem{Linac}
N. Iida, M. Kikuchi, K. Furukawa, M. Ikeda, K. Kakihara, T. Kamitani, Y. Kobayashi, T. Mitsuhashi, Y. Ogawa, M. Satoh, T. Suwada, M. Tawada and K. Yokoyama,
\newblock ``NEW BEAM TRANSPORT LINE FROM LINAC TO PHOTON FACTORY IN KEK",
\newblock Proceedings of EPAC, Edinburgh, pp. 1505-1508, paper TUPLS010, 2006.

\bibitem{Kobayashi}
Y. Kobayashi, A. Ueda, and T. Mitsuhashi,
\newblock ``INJECTION PERFORMANCE WITH A TRAVELING WAVE KICKER MAGNET SYSTEM AT THE PHOTON FACTORY STORAGE RING",
\newblock Proceedings of the 2003 Particle Accelerator Conference, pp. 3204-3206, paper RPPG013, 2003.

% Chapter 2
% Beam Dynamics & Injection Scheme
\bibitem{Hochi}
發知英明,
\newblock ``大強度陽子リングのビーム力学",
\newblock 高エネルギー加速器セミナーOHO, 2010.

\bibitem{Courant_inv}
E. Courant and H. Snyder,
\newblock ``THEORY OF THE ALTERNATING-GRADIENT SYNCHROTRON",
\newblock Annals of Physics, 281, pp. 60-408, 2000.

% Chapter 3
% Injection Scheme
\bibitem{Wrulich}
A. Wrulich,
\newblock ``SINGLE BEAM LIFETIME",
\newblock CERN-94-01, pp. 409, 1994.

\bibitem{kicker_inj}
Gottfried Mülhaupt,
\newblock ``SYNCHROTRON RADIATION SOURCES A PRIMER",
\newblock World Scientific, Singapore, Chapter 3, 1994.

\bibitem{injection_nsls}
G.M. Wang, W.X. Cheng, X. Yang, J. Choi and T.shaftann,
\newblock ``STORAGE RING INJECTION KICKERS ALIGNMENT OPTIMIZATION IN NSLS-2",
\newblock Proceedings of 8th International Particle Accelerator Conference, pp. 4683-4685, paper THPVA095, 2017.

% Chapter 4
% Beam based measurement of injection parameters
\bibitem{beam_based_meas}
K. Hirano, K. Harada, N. Higashi, S. Nagahashi, A. Ueda, T. Obina, R. Takai, H. Takaki and Y. Kobayashi,
\newblock ``BEAM BASED MEASUREMENT OF INJECTION PARAMETERS AT KEK-PF",
\newblock Proceedings of 9th International Particle Accelerator Conference, pp. 4152-4154, paper thpmf042, 2018.

\bibitem{beam_based_meas2}
K. Hirano, K. Harada, N. Higashi, S. Nagahashi, A. Ueda, T. Obina, R. Takai, H. Takaki and Y. Kobayashi,
\newblock ``KEK-PF におけるビームベース測定を用いた入射効率改善のための研究",
\newblock Proceedings of the 15th Annual Meeting of Particle Accelerator Society of Japan, pp. 171-175, paper FROL05, 2018.

\bibitem{Libera}
Libera Brilliace plus,
\newblock https://www.i-tech.si/accelerators-instrumentation/libera-brilliance-plus/

\bibitem{PF_kicker}
A. Ueda, T. Ushiku and T. Mitsuhashi,
\newblock ``CONSTRUCTION OF TRAVELLING WAVE KICKER MAGNET AND PULSE POWER SUPPLY FOR THE KEK-PHOTON FACTORY STORAGE RING",
\newblock Proceedings of the 2001 Particle Accelerator Conference, pp. 4050-4054, 2001.

\bibitem{Kuboki}
久保木浩功,
\newblock ``陽子ビームモニター",
\newblock 高エネルギー加速器セミナーOHO, 2018.

\bibitem{LINAC_timing}
F. Miyahara, H. Kaji, H. Katagiri, M. Satoh, H. Sugimura, M. Satoh, X. Zhou, R. Zhang, K. Furukawa, T. Matsumoto, T. Miura, M. Yoshida,
S. Kusano, T. Kudou, H. Kumano, T. Oofusa and H. Saotome,
\newblock ``TIMING SYSTEM FOR THE KEK e+/e- INJECTOR LINAC",
\newblock Proceedings of the 15th Annual Meeting of Particle Accelerator Society of Japan, pp. 544-548, paper WEP082, 2018.

% Chapter 5
% Injection Simulation at KEK-PF
\bibitem{Kobayashi_oho}
小林幸則,
\newblock ``電子ストレージリング",
\newblock 高エネルギー加速器セミナーOHO, 1993.

\bibitem{castro}
P. Castro, J. Borer, A. Burns, G. Morpurgo and R. Schmidt,
\newblock ``BETATRON FUNCTION MEASUREMENT AT LEP USING THE BOM 1000 TURNS FACILITY",
\newblock Proceedings of the 2001 Particle Accelerator Conference, pp. 2103-2105, 1993.

\bibitem{LOCO}
J. Safranek,
\newblock ``EXPERIMENTAL DETERMINATION OF STORAGE RING OPTICS USING ORBIT RESPONSE MEASUREMENTS",
\newblock Nuclear Instruments and Methods in Physics Research Section A, 388, 27-36, 1997.

\bibitem{SAD}
SAD,
\newblock http://acc-physics.kek.jp/SAD/

\bibitem{takagi}
高木宏之,
\newblock 博士論文``電子蓄積リングにおけるパルス6極電磁石を用いた入射システムの開発研究"

% Chapter 6
% On line adjustment of accelerator
\bibitem{RCDS}
Xiaobiao Huang, Jeff Corbett, James Safranek and Juhao Wu,
\newblock `AN ALGORITHM FOR ON-LINE OPTIMIZATION OF ACCELERATORS",
\newblock Nuclear Instruments and Methods in Physics Research, 726, 77–83, 2013.

\bibitem{GA_method}
K. Tian, J. Safranek and Y. Yan,
\newblock ``MACHINE BASED OPTIMIZATION USING GENETIC ALGORITHMS IN A STORAGE RING",
\newblock Phys. Rev. ST Accel. Beams, 17, 020703, 2014.

\bibitem{conj}
William H. Press, Saul A. Teukolsky, William T. Vetterling and Brian P. Flannery,
\newblock ``NUMERICAL RECIPES",
\newblock Cambridge University Press, Third edition, Chapter 10, 515-520, 2007.
% Appendix
% Injection scheme
\bibitem{kicker_jump}
M. Tobiyama, E. Kikutani, J. W. Flanagan and S. Hiramatsu, 
\newblock ``BUNCH BU BUNCH FEEDBACK SYSTEMS FOR THE KEKB RINGS",
\newblock Proceedings of the 2001 Particle Accelerator Conference, pp. 1246-1248, 2001.

\bibitem{injection_SLS}
A. Saa Hernandez and M. Aiba,
\newblock ``INVESTIGATION OF THE INJECTION SCHEME FOR SLS 2.0",
\newblock Proceedings of 6th International Particle Accelerator Conference, pp. 1720-1723, paper TUPJE046, 2015.

\bibitem{harada_pri}
K. Harada,
\newblock Private Communication.

\end{thebibliography}

%******************************************************************************

\end{document}
