\documentclass[dvipdfmx,12pt]{beamer}
\usepackage{bxdpx-beamer}
\usepackage{pxjahyper}
\usepackage{minijs}
\renewcommand{\kanjifamilydefault}{\gtdefault}

\usetheme{Madrid}
\usefonttheme{professionalfonts}
\usecolortheme{whale}
\setbeamertemplate{frametitle}[default][center]
\setbeamertemplate{navigation symbols}{}
%\setbeamercovered{transparent}



\author{平野広太}
\title{KEK-PFにおける\\入射効率改善のための研究}
\institute{放射光物理研究室}
\date{\today}

\begin{document}

\maketitle

\begin{frame}\frametitle{Contents}
\tableofcontents
\end{frame}

\section{はじめに}
\subsection{研究背景と目的}
\begin{frame}\frametitle[はじめに]{研究背景と目的}
\begin{block}{Beamerのよいところ}
\begin{itemize}
\item 論文・レジメの再利用が容易 
\item レイアウトの微調整は\LaTeX に任せることができる 
\item 数式がきれい 
\end{itemize}
\end{block}
\end{frame}

\subsection{キッカー入射}
\begin{frame}\frametitle{入射}
\begin{block}{Beamerのよいところ}
\begin{itemize}
\item 論文・レジメの再利用が容易 
\item レイアウトの微調整は\LaTeX に任せることができる 
\item 数式がきれい 
\end{itemize}
\end{block}
\end{frame}

\section{入射パラメータの測定}
\begin{frame}\frametitle{何故Beamerを使うのか?}
\begin{block}{Beamerのよいところ}
\begin{itemize}
\item 論文・レジメの再利用が容易 
\item レイアウトの微調整は\LaTeX に任せることができる 
\item 数式がきれい 
\end{itemize}
\end{block}
\end{frame}

\subsection{ビームベース測定の概要}
\begin{frame}\frametitle{何故Beamerを使うのか?}
\begin{block}{Beamerのよいところ}
\begin{itemize}
\item 論文・レジメの再利用が容易 
\item レイアウトの微調整は\LaTeX に任せることができる 
\item 数式がきれい 
\end{itemize}
\end{block}
\end{frame}

\subsection{入射ビームの測定}
\begin{frame}\frametitle{何故Beamerを使うのか?}
\begin{block}{Beamerのよいところ}
\begin{itemize}
\item 論文・レジメの再利用が容易 
\item レイアウトの微調整は\LaTeX に任せることができる 
\item 数式がきれい 
\end{itemize}
\end{block}
\end{frame}

\subsection{キッカー電磁石・セプタム電磁石}
\begin{frame}\frametitle{何故Beamerを使うのか?}
\begin{block}{Beamerのよいところ}
\begin{itemize}
\item 論文・レジメの再利用が容易 
\item レイアウトの微調整は\LaTeX に任せることができる 
\item 数式がきれい 
\end{itemize}
\end{block}
\end{frame}

\section{入射シミュレーション}
\begin{frame}\frametitle{何故Beamerを使うのか?}
\begin{block}{Beamerのよいところ}
\begin{itemize}
\item 論文・レジメの再利用が容易 
\item レイアウトの微調整は\LaTeX に任せることができる 
\item 数式がきれい 
\end{itemize}
\end{block}
\end{frame}

\subsection{キッカーバンプ・マッチング}
\begin{frame}\frametitle{何故Beamerを使うのか?}
\begin{block}{Beamerのよいところ}
\begin{itemize}
\item 論文・レジメの再利用が容易 
\item レイアウトの微調整は\LaTeX に任せることができる 
\item 数式がきれい 
\end{itemize}
\end{block}
\end{frame}

\subsection{入射振動・入射効率}
\begin{frame}\frametitle{何故Beamerを使うのか?}
\begin{block}{Beamerのよいところ}
\begin{itemize}
\item 論文・レジメの再利用が容易 
\item レイアウトの微調整は\LaTeX に任せることができる 
\item 数式がきれい 
\end{itemize}
\end{block}
\end{frame}

\section{入射調整}
\begin{frame}\frametitle{何故Beamerを使うのか?}
\begin{block}{Beamerのよいところ}
\begin{itemize}
\item 論文・レジメの再利用が容易 
\item レイアウトの微調整は\LaTeX に任せることができる 
\item 数式がきれい 
\end{itemize}
\end{block}
\end{frame}

\section{まとめ}
\begin{frame}\frametitle{何故Beamerを使うのか?}
\begin{block}{Beamerのよいところ}
\begin{itemize}
\item 論文・レジメの再利用が容易 
\item レイアウトの微調整は\LaTeX に任せることができる 
\item 数式がきれい 
\end{itemize}
\end{block}
\end{frame}


\end{document}