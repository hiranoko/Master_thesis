\documentclass{jsarticle}
\usepackage[dvipdfmx]{graphicx}
\usepackage{siunitx}   % SI単位系

\usepackage[top=34truemm,bottom=30truemm,left=30truemm,right=30truemm]{geometry} 

\begin{document}
\pagestyle{empty}

\begin{center}
\fontsize{17.5pt}{0pt}\selectfont
KEK-PFにおける入射効率改善のための研究
\vspace{5mm}
\end{center}

\begin{flushleft}
\fontsize{11.5pt}{0pt}\selectfont
  平野広太    M170458    放射光物理研究室
\vspace{3mm}
\end{flushleft}

\fontsize{11pt}{18pt}\selectfont
近年、高エネルギー加速器研究機構(KEK) における蓄積リング型放射光源(PFリング) では2011年の東日本震災によって生じた、ビーム輸送路の電磁石やセプタム電磁石等のアライメント誤差などが原因で入射効率の低下が問題となっている。特にセプタム電磁石は真空漕に入っていることやターゲット座がないため、再据付されておらず正確な入射ビームの位置が特定されていない。また入射効率低下の原因に、各キッカー電磁石の振幅精度や励磁タイミングの不揃いなどが考えられている。

本研究は、高エネルギー加速器研究機構(KEK) における蓄積リング型放射光源(PFリング) の入射効率を改善するため、現在の入射パラメータを実験的に明らかにすると共に、問題の解決を目的とする。

入射パラメータの測定はビームベースで行った。測定は直線部両端のダクトのビーム位置モニター(BPM) 電極に検出回路を接続してドリフトスペースを挟んだ 2 点の位置情報から、直線部中心の位相空間情報を計算し、上流に転送する方法で行った。本手法を用いると磁場測定だけでは観測されない、あらゆる誤差を含んだ現実の加速器を対象とする、リング内の電磁石から制御系までの環境を含んだ効果を調べることが可能となる。また、キッカー電磁石の励磁タイミングを掃引して応答を調べることでパルス波形が得られる。測定したパラメータは入射シミューレーションに反映して、原因の探索と最適化を実施した。最適化は入射効率の改善と蓄積ビームの振動の抑制を目標とする。入射効率は加速器シミュレーションコードSAD で多粒子追跡を行い評価した。また蓄積ビームの振動抑制の評価はキッカー電磁石の蹴り角に偏差を導入して生じるベータトロン振動の大きさを評価した。その結果、以下のような知見が得られた。

\begin{enumerate}
\item 入射効率低下の原因は、セプタム電磁石の位置が変わり入射振幅が大きくなったこと
\item 入射振幅の増加に伴って、垂直振動を含む入射は効率低下の原因となること、またセプタムの蹴り角が足りていないこと
\item キッカーバンプマッチングの計算から、キッカー電磁石の励磁タイミングを全てピーク位置で揃えるより、タイミングを変えることで振動を抑制できること
\end{enumerate}

以上のように、本研究により、入射効率低下の原因を調べ、蓄積ビームの振動の抑制の要件を満たした解はビームスタディでも有効であることを確認した。


\end{document}
