\documentclass{jsarticle}
\usepackage[dvipdfmx]{graphicx}
\usepackage{siunitx}   % SI単位系

\usepackage[top=30truemm,bottom=30truemm,left=30truemm,right=30truemm]{geometry} 

\begin{document}
\pagestyle{empty}

\begin{center}
\vspace{4mm}
\fontsize{16pt}{0pt}\selectfont
KEK-PFにおける入射効率改善のための研究
\vspace{5mm}
\end{center}


\fontsize{11pt}{0pt}\selectfont
\begin{minipage}[c]{38mm}
平野広太
\end{minipage}
\begin{minipage}[c]{32mm}
M170458
\end{minipage}
\begin{minipage}[c]{80mm}
放射光物理研究室
\end{minipage}
\vspace{5mm}

\fontsize{10.5pt}{18pt}\selectfont
% PFリングについて、入射の要件
高エネルギー加速器研究機構 (KEK)にあるPhoton Factory (PF)では、電子加速器から発生する放射光を使って、物性・生命の構造から機能発現の仕組みの研究が行われている。加速器で蓄積された電子ビームは、残留ガスとの散乱や量子寿命、ビーム内散乱など様々な要因で失われてゆく。これを回復するため、線形加速器 (LINAC)から供給される電子ビームを注ぎ足す必要がある。LINACで作られた電子ビームはビーム輸送路を通り蓄積リングまで輸送される。この輸送路からリングへのビームを受け渡しを入射と呼ぶ。入射では輸送によるビームロスを最小にして、目的の位相空間上へ輸送されることが要求される。

% キッカー入射について、電磁石の要求
PFではセプタムとキッカーを用いた入射方式を採用している。輸送路終端には、入射ビームの向きを蓄積リングの軌道に揃える為、セプタム電磁石が設置される。セプタム板では入射ビームと蓄積ビームの間を渦電流の効果で仕切り、リング側に磁場が漏れない工夫がなされる。入射ビームを蓄積リングのダクトに納めるには、入射ビームが蓄積リングに入った後、軌道を曲げ中心に近づける必要がある。その際、蓄積ビームも軌道を曲げられるため、入射部分の上流で予め蹴っておくという操作がなされる。結果として蓄積ビームの軌道は、入射点付近で入射の瞬間のみ入射点側に移動する。これをパルスバンプと呼ぶ。パルスバンプを作る電磁石をキッカーと呼び、リングの周回周期時間の半正弦波波形で励磁され、入射ビームはそのピークのタイミングに合わせられる。パルス幅が短いため、高電圧、大電流の電源が必須となる。

% キッカー入射の問題
入射時のキッカー誤差による、蓄積ビームの振動は放射光の揺らぎとなり、入射ビームの損失は放射線の発生につながるため、どちらも抑制する必要がある。入射中は実験を休止するという前提で作られたPFでは、入射点が遠い為に大きなキッカーバンプが必要であり、その為の特殊なパルス電源では複数台のキッカーの波形やタイミングを完全に一致させて誤差を現代的なレベルまで低減させることが極めて困難である。また、パルス磁石の磁場は環境やダクトに依存し、実験室での準備測定そのままではない。さらに、セプタムは1980年代製造である為、真空内外の位置調整が極めて困難である上に、2011年の震災で正確な据付座標を失った。

% 本研究の目的、研究手法、結果、今後の課題
そこで本研究は、電子ビームを直接測定する手法により、現在の正確な入射状況を把握した上で改善することを目的に入射パラメータの測定と最適化実験を実施した。測定では、パルス電磁石の波形や蹴り角、タイミングと入射点の位置情報の推定を行った。ドリフト空間を挟んだ2点でのビームの位置情報があれば、ドリフト空間中心でのビームの位相空間が得られる。そこから軌道を上流に遡れば、任意の位置の軌道情報が得られる。実際に、直線部両端の検出回路からの生信号を、独自に開発した手法で解析し、入射パラメータ等の推定を行った。最適化実験では、現在の入射状況を反映した上で入射パラメータの最適化のシミュレーションを行い、ビームを使って最適化実験を進めるプログラムも開発した。ビームを使って最適化実験を行った結果、シミュレーション通りに最適化できることが分かり、入射タイミング以外のタイミングを含む、蓄積ビーム全体にわたって誤差による残留振動を小さくしながら、入射率を落とさないパラメータを発見した。

%     \begin{figure}[htbp]
%        \centering
%        \includegraphics[width=150mm]{./fig1.png}
%        \caption{最適化実験による蓄積ビームの振動抑制}
%    \end{figure}

\end{document}
