\documentclass{jsarticle}
\usepackage[dvipdfmx]{graphicx}
\usepackage{siunitx}   % SI単位系

\usepackage[top=34truemm,bottom=30truemm,left=30truemm,right=30truemm]{geometry} 

\begin{document}
\pagestyle{empty}

\begin{center}
\fontsize{17.5pt}{0pt}\selectfont
KEK-PFにおける入射効率改善のための研究
\vspace{5mm}
\end{center}

\begin{flushleft}
\fontsize{11.5pt}{0pt}\selectfont
  平野広太    M170458    放射光物理研究室
\vspace{3mm}
\end{flushleft}

\fontsize{11pt}{18pt}\selectfont
本論文では高エネルギー加速器研究機構(KEK) における蓄積リング型放射光源(PFリング) で実施した入射に関する研究の成果を報告する。研究の主な目的は入射効率の改善、つまりビーム輸送におけるロスを改善することにある。PFリングではキッカー電磁石に設定値から振幅や励磁タイミングの誤差を含むことやマルチターンキック入射の影響のため、入射時の蓄積ビームの振動や入射効率低下の問題があった。そこに2011年の東日本大震災によるビーム輸送路の電磁石やセプタム電磁石等のアライメント誤差などが原因で、入射効率の低下が問題となっている。特に入射セプタム電磁石は真空槽内にありターゲット座がないため再配置されていない。

蓄積リングを周回するビーム電流値は、ガス散乱や量子寿命、ビーム内散乱によって時間と共に減ってゆく。これを回復するため入射器から供給されるビームは輸送路を通してリングへ注ぎ足される。この輸送路からリングへの受け渡しを入射と呼ぶ。PF リングではキッカーとセプタムのパルス電磁石を用いた入射方式を採用している。セプタム電磁石はビーム輸送路の終端部に設置され、入射ビームのみの軌道の向きを変えて蓄積ビームの設計軌道とそろえるために用いられる。入射ビームの向きだけ曲げ、蓄積ビームに影響を与えないようにするため、セプタム壁で輸送路とリングを区切り渦電流の効果を利用して磁場を遮蔽している。キッカー電磁石は入射点にバンプを形成するため、入射点より上流に2 台、下流に2 台が設置される。上流のキッカー電磁石は、リングを周回するビームを入射路側へ寄せて入射振動の振幅を小さくする働きをする。下流のキッカー電磁石は上流のキッカー電磁石と組み合わせて閉じたバンプを形成する。入射ビームは、セプタム壁を挟んだ入射ビームと蓄積ビームの間の距離が初期振幅として、振動しながらリングを周回する。振動は放射減衰によって決まる時間のスケールで減衰して蓄積ビームに捕獲される。しかし現実のパルス電磁石では、設定値から振幅やタイミングの誤差を含む。またパルスの立ち上がり時間と立ち下がり時間、パルス波形が揃わないと設定タイミングの前後でマッチングが崩れる。他にもバンプ内に含まれる多極磁場成分やフリンジの影響など非線形要素が誤差となり、閉じたバンプにならない。PF リングでは、特にキッカー電磁石のパルス長の違いが原因となって、入射時の蓄積ビームの振動や入射効率低下の問題が存在した。

本研究では、はじめにビームベースでの入射パラメータの特定を行った。測定は長直線部両端のBPM電極にビーム位置検出回路を接続して、ドリフトスペースを挟んだ2 点の位置情報から直線部中心の位相空間情報を計算し、上流に転送する方法で行った。その結果、入射ビームの位相空間情報及びキッカー電磁石のパルス波形、セプタム電磁石の応答などのパラメータを得た。つぎに測定で求めた入射パラメータを用いて、計算とビーム測定による最適化を実施した。入射パラメータの計算では入射効率の改善と蓄積ビームの振動の抑制の入射要件を満たしたパラメータの組み合わせを調べる。入射効率は加速器シミュレーションSAD で多粒子追跡を行い評価した。また蓄積ビームの振動抑制の評価はキッカー電磁石の蹴り角に偏差を導入して生じるベータトロン振動の大きさを周期構造で評価した。以上より、計算から求めた入射の要件を満たした解はビームスタディでも有効であることを確認した。

\end{document}
