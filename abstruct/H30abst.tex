\documentclass{jsarticle}
\usepackage[dvipdfmx]{graphicx}
\usepackage{siunitx}   % SI単位系

\usepackage[top=30truemm,bottom=30truemm,left=30truemm,right=30truemm]{geometry} 

\begin{document}
\pagestyle{empty}

\begin{center}
\vspace{4mm}
\fontsize{17.5pt}{0pt}\selectfont
KEK-PFにおける入射効率改善のための研究
\vspace{5mm}
\end{center}


\fontsize{11.5pt}{0pt}\selectfont
\begin{minipage}[c]{38mm}
平野広太
\end{minipage}
\begin{minipage}[c]{32mm}
M170458
\end{minipage}
\begin{minipage}[c]{80mm}
放射光物理研究室
\end{minipage}
\vspace{5mm}

\fontsize{10.5pt}{18pt}\selectfont
% PFリングについて、入射の要件
高エネルギー加速器研究機構 (KEK)にあるPhoton Factory (PF)では、電子加速器から発生する放射光を使って、物性・生命の構造から機能発現の仕組みの研究が行われている。放射光とは電子ビームが加速を受けた際に放射する電磁波であり、PFは高エネルギーの電子ビームを蓄積して、加速器中を周回させる蓄積リングとして活用されている。加速器に蓄積された電子ビームは、周回中に残留ガスとの散乱や量子寿命、ビーム内散乱など様々な要因で失われてゆく。これを回復するため、線形加速器 (LINAC)から供給される電子ビームを注ぎ足す必要があり、LINACで作られた電子ビームはビーム輸送路を通り蓄積リングまで輸送される。この輸送路からリングへのビームを受け渡しを入射と呼ぶ。入射では輸送によるビームロスを最小にして、目的の位相空間上へ輸送されることが要求される。また、Liouvilleの定理よりリングの設計軌道をまわる粒子には、重ならないようにして、ダクトに入射しないといけない。

% キッカー入射について、電磁石の要求
PFではセプタム電磁石とキッカー電磁石を用いた入射方式を採用している。輸送路終端には、入射ビームの向きを蓄積リングの軌道に揃える為のセプタム電磁石が存在し、セプタム(横隔膜の意)板で入射ビームと蓄積ビームの間を仕切り、パルス的に励磁することで、板上の渦電流で蓄積側に磁場が漏れない工夫がなされている。入射ビームを蓄積リングのダクトに納めるには、入射ビームが蓄積リングに入った後で軌道を曲げて中心に近づける必要があるが、その際、同時に既にある蓄積ビームも必ず影響を受ける。そこで、蓄積リングに対してはその影響が打ち消されるように、入射部分の上流で予め蹴っておくという操作がなされる。結果として蓄積ビームの軌道は、入射点付近で入射の瞬間のみ入射点側に移動する、ということになり、それを局所パルスバンプと呼んでいる。パルスバンプを作る電磁石をキッカーと呼び、リングの周回周期(PFの場合$\neq 1.6 \si{s}$)の半正弦波波形で励磁され、入射ビームはそのピークのタイミングに合わせられる。パルス幅が短いため、高電圧、大電流の電源が必須となる。

% キッカー入射の問題
入射時のキッカー誤差による、蓄積ビームの振動は放射光の揺らぎとなり、入射ビームの損失は放射線の発生につながるため、どちらも抑制する必要がある。常時入射を行い、蓄積電流を一定に保つというトップアップが主流となった現在、入射中は実験を休止するという前提で作られたPFでは、入射点が遠い為に大きなキッカーバンプが必要であり、その為の特殊なパルス電源では複数台のキッカーの波形やタイミングを完全に一致させて誤差を現代的なレベルまで低減させることが極めて困難である。また、パルス磁石の磁場は環境やダクトに依存し、実験室での準備測定そのままではない。さらに、セプタムは1980年代製造である為、真空内外の位置調整が極めて困難である上に、2011年の震災で正確な据付座標を失った。

% 本研究の目的
そこで本研究は、電子ビームを直接測定する手法により、現在の正確な入射状況を把握した上で改善することを目的に入射パラメータの測定と最適化実験を実施した。

% 研究手法、結果、今後の課題
測定では、電子ビームを使ってパルス電磁石の波形や蹴り角、タイミングと入射点の位置情報の推定を行った。蓄積リングの間に何もない空間を挟んだ2点でのビームの位置情報があれば、その中央でのビームの位置と向きが得られる。そこから軌道を上流に遡れば、任意の位置の軌道情報が得られる。実際に、直線部両端のビーム位置モニタの電極からの生の電圧信号を、独自に開発した手法で解析し、微小電荷に対する正確な瞬時測定を可能にして入射パラメータ等の推定を行った。測定したパラメータを用いて、入射パラメータの最適化のシミュレーションを行い、ビームスタディで測定を行いながら最適化を進めるプログラムも開発した。実際にビームを使って最適化実験を行った結果、ほぼシミュレーション通りに最適化できることが分かり、実際に、入射タイミング以外のタイミングを含む、蓄積ビーム全体にわたって誤差による残留振動を小さくしながら、入射率を落とさないパラメータを発見した。

     \begin{figure}[htbp]
        \centering
        \includegraphics[width=150mm]{./fig1.png}
        \caption{最適化実験による蓄積ビームの振動抑制}
    \end{figure}

\end{document}
